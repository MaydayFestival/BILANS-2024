\documentclass[12pt,a4paper,draft]{report}
\usepackage[utf8]{inputenc}
\usepackage[french]{babel}
\usepackage{microtype}
\usepackage[T1]{fontenc}
\usepackage{amsmath}
\usepackage{amsfonts}
\usepackage{amssymb}
\usepackage{graphicx}
\usepackage{lmodern}
\usepackage{array}
\usepackage{multirow}

\linespread{1.2}

\usepackage{geometry}
\geometry{hmargin=2cm,vmargin=1.5cm}

\usepackage{soul} \usepackage{color} \newcommand{\hilight}[1]{\colorbox{yellow}{#1}}

\title{Bilan de la 6e édition du Festival Mayday (2024)}
\author{Association M-Tech}
\date{Vendredi 31 mai 2024}

\makeindex

\begin{document}
\maketitle

\begin{abstract}
Le Mayday Festival est une manifestation culturelle. Mêlant musique et sensibilisation aux transitions socio-environnementales, ses objectifs sont :
\begin{itemize}
\item Ouverture de l'accès à la culture à toutes et tous,
\item Sensibilisation du plus grand nombre aux transitions environnementales et sociétales,
c\item Être un laboratoire d'incubation, chef de fil en matière d'événementiel éco-responsable,
\item Mettre en avant des artistes locaux, émergents et étudiants
\end{itemize}
Dans le but d'être ouvert à plus de public et de permettre le passage de davantage d'artistes, la 6e édition du festival se déroulait sur 2 jours.Cette évolution ambitieuse, mais cohérente avec les objectifs, fera l'objet d'une analyse dans ce rapport. \\
Plusieurs périodes ont composé l'organisation et le déroulement du festival. Ces séquences seront détaillées dans le rapport avant les bilans et la conclusion de cette édition.
\end{abstract}

\tableofcontents

\newpage

Introduction
\begin{itemize}
\item Nom du festival, dates et lieu
\item Objectifs initiaux du festival
\end{itemize}

Bilan de la fréquentation
\begin{itemize}
\item Nombre total de visiteurs
\item Répartition de la fréquentation
\item Analyse de la fréquentation
\end{itemize}

Bilan financier
\begin{itemize}
\item Recettes
\item Dépenses
\item Résultat financier
\end{itemize}

Bilan artistique et programmation
\begin{itemize}
\item Programmation
\item Analyse de la programmation
\end{itemize}

Bilan de la communication et du marketing
\begin{itemize}
\item Moyens de communication utilisés
\item Analyse de la communication
\end{itemize}

Bilan de l'organisation et de la logistique
\begin{itemize}
\item Équipe
\item Logistique
\item Analyse de l'organisation
\end{itemize}

Bilan environnemental et social
\begin{itemize}
\item Impact environnemental
\item Impact social
\end{itemize}

\part{Gestion de projet (août 2023 - avril 2024)}
\chapter{Membres de l'organisation}
\section{Contexte}
Les années marquées par la pandémie "Covid-19" ont eu un fort impact sur la transmission des compétences, connaissances et réseaux des membres d'associations. Le Festival Mayday était porté depuis 2016 par le même noyau dûr, puis repris par une équipe majoritairement novice pour la 5e édition (2023). L'objectif de cette année était notamment de capitaliser sur les connaissances et réseaux restaurés sur les années précédentes. 

\section{Méthodologie}
\subsection{Recrutement}
De nombreux membres de l'organisation ayant fini leurs études, l'équipe d'organisation s'est largement renouvelée en fin 2023.
Le recrutement s'est fait via plusieurs méthodes : 
\begin{itemize}
\item Réseau proche
\item Annonce en ligne
\end{itemize}

\subsubsection{Réseau proche}
Au travers d'expériences passées, les membres de l'organisation avaient pu interagir avec d'autres étudiants. De nombreux profils avaient étés pré-identifiés, cependant l'ampleur du projet, l'investissement nécessaires et les attendus sur les postes ouverts peuvent être des facteurs limitant.

\subsection{Pôles}
\subsubsection{Enjeux}
L'organisation du festival se construit selon plusieurs enjeux : 
\begin{itemize}
\item Coordination du projet,
\item Sécurité de l'événement,
\item Déclaration administratives,
\item Programmation événementielle,
\item Matériel,
\item Électricité,
\item Sensibilisation,
\item Bilan carbone,
\item Communication,
\item Partenariats,
\item Gestion des bénévoles,
\item Gestion financière,
\item Restauration sur site (nourriture et boisson). \\
\end{itemize}

Une organisation en pôles a donc été mise en place, avec 2 type de pôle : 
\begin{itemize}
\item Orienté
\item Transverse/Support
\end{itemize}

\subsubsection{Version finale}
Plusieurs essais ont étés nécessaires, avec de nombreux ajustements. Nous avons obtenu cette liste (dans l'ordre alphabétique) : 
\begin{itemize}
\item Administratif
\item Artistes
\item Bar
\item Bénévoles
\item Communication
\item Innovation écologique et sociale
\item Logistique
\item Partenariats
\item Presse
\item Restauration
\item Sécurité technique prévention
\item Transverse
\item Trésorerie
\end{itemize}

\subsection{Interactions}
Pour organiser cette 6e édition, de nombreux échanges entre les différentes entités crées devait se faire. Nous pouvons les catégoriser en plusieurs types, à voir ci-dessous.

\subsubsection{Administratif}
Dans le cadre de déclarations (sécurité notamment), de nombreux échanges étaient nécessaires pour que des informations à jour soient communiqués. Ils comprenaient plusieurs volets :
\begin{itemize}
\item Implantation (plans, simulations),
\item Besoins spécifiques,
\item Risques spécifiques,
\item Utilisations particulières de matériel (non conforme à l'usage préconisé),
\item Installation de CTS (Chapiteaux, tentes, structures),
\item Plans alternatifs (météo)
\item Suivi des dates limites de dépôts, échanges avec les autorités, suivi des nouvelles réglementations.
\end{itemize}

\subsubsection{Communication et partenariats}
Dans le but de mettre en avant le travail fait bénévolement par ses membres, plusieurs personnes avaient pour mission de communiquer sur :
\begin{itemize}
\item l'événement,
\item les engagements,
\item la gestion de projet,
\item les manifestations annexes (participations à des concours, tenue de buvettes, ...)
\item ...
\end{itemize}

Pour cela, des échanges réguliers et la présence de l'équipe de communication en réunion hebdomadaire permettaient une remontée conforme de nos avancées et de la direction prise par le projet auprès du grand public.

Sur le plan partenarial (public professionnel, politiques, administrations), la mise en avant des enjeux de la manifestation, ainsi que les engagements humains derrière la façade festive et engagée. Elle se faisait notamment au travers d'une couverture photo/vidéo de nos interventions (séminaires, demandes de subventions, réunions).


\subsubsection{Montage de projet}
De manière plus générale, l'ensemble du montage du projet nécessitait des échanges globaux. Des point hebdomadaires étaient organisés, permettant que l'ensemble de l'équipe ait le même niveau d'information.


\section{Membres de l'équipe "Mayday 2024"}
Cette année, l'équipe du Mayday était composée par :
\begin{center}
\begin{tabular}{| m{3cm} | m{9cm} |}
\hline
Pôle & Prénom \\ 
 \hline\hline\\
Transverse & \textbf{Louis Delignac}\\ 
\hline\\
\multirow{2}{5em}{Sécurité technique prévention} & \textbf{Sacha Duperret} \\ 
& Antoine Philippeau, Ebène Bilé, Maël Avadian, Flavio Arella\\  
 \hline\\
 \multirow{2}{5em}{Innovation écologique et sociale} & \textbf{Baptiste Royau} \\ 
& Katarina Montenier, Manon Bénard, Eléa Peynaud, Leelou Cateigbou\\
 \hline\\
  \multirow{2}{5em}{Trésorerie} & \textbf{Aurélien Gauthier} \\ 
& Victor Lamare\\
 \hline\\
 \multirow{2}{5em}{ Artistes} & \textbf{Leïla Karim} \\
& Lucile Guerin\\
 \hline\\
\multirow{2}{5em}{Bénévoles} & \textbf{Nabil Es-Selymy} \\ 
& Anessa Diallo\\
 \hline\\
\multirow{2}{7em}{Communication} & \textbf{Alice Conti} \\ 
& Salma Rabion, Nassim Sellal, Julie Fourneyron\\
 \hline\\
 \multirow{1}{5em}{Logistique} & \textbf{Arthur Vienot} \\ 
 \hline\\
 \multirow{1}{6em}{Restauration} & \textbf{Lucas Brouet} \\ 
 \hline
\end{tabular}
\end{center}

\part{Installation et mise en place (10 mai - 17 mai 2024)}
\chapter{Livraisons}
Du vendredi 10 mai au vendredi 17 mai, le festival fut mis en place. 
\section{Méthodologie}


\section{Opérationnel}
\subsection{Sécurité}
Plusieurs enjeux de sécurité coexistaient durant les temps d'installation et de mise en place du festival : 
\begin{itemize}
\item Sécurisation des livraisons,
\item Protection des usagers classiques du campus,
\item Stockage du matériel (nuit et intempéries).
\end{itemize} 


Le festival se déroulant sur le campus universitaire de sciences durant une période d'activité, l'organisation du festival devait s'adapter : 
\begin{itemize}
\item Niveau de bruit faible pendant les horaires d'enseignements ainsi que la nuit (voisinage),
\item Maintenir les accès et signaler les déviations, notamment PMR et vélo, entre les différents bâtiments et pour traverser le campus,
\item Partager les espaces intérieurs avec les utilisateurs habituels (services, enseignants, associations étudiantes, ...).
\end{itemize}

Pour cela, un étalement de l'installation a été mis en place. Habituellement commençant au milieu de la semaine, les livraisons ont commencées dès le vendredi précédent. Cela permettait particulièrement de mieux gérer les flux d'usagers du campus.

Par ailleurs, le soutien de l'université de Bordeaux au travers de son service scolarité du collège sciences et technologies était ici primordial. Il a permis le déplacement de nombreux enseignements à distance des espaces "à risques de nuisances" vers des lieux plus calme, à proximité.

Un arrêté du président de l'UB prévoyait, dès le début de la semaine, la fermeture de plusieurs axes de circulation (piéton, vélo et VL). Cela permettait de :
\begin{itemize}
\item Mettre en place un barriérage de sécurité autour du site (installation progressive suivant l'installation),
\item Ouvrir des voies d'accès sécurisée depuis et vers les bâtiments ainsi que les sorties du campus,
\item Faciliter la sécurisation des livraisons, dont les manœuvres pouvaient avoir lieu dans un environnement contrôlé,
\item Concourir à la réussite du travail des agents de sécurité en mission de gardiennage de nuit.
\end{itemize}



\section{Bilan}

\part{Opérationnel (17 mai - 18 mai 2024)}
\section{Sécurité}


\part{Désinstallation et rangement (19 mai - 7 juin 2024)}

\part{Bilan moral}
\chapter{Partenaires}

\part{Bilan financier}
\chapter{Analyse financière - passage à 2 jours}

\subsection{subsection}
\subsubsection{subsubsection}
\paragraph{paragraph}
\subparagraph{subparagraph}

\end{document}