\documentclass[12pt,a4paper]{report}
\usepackage[utf8]{inputenc}
\usepackage[french]{babel}
\usepackage[hidelinks]{hyperref}
\usepackage{microtype}
\usepackage{fontenc}
\usepackage[scaled]{helvet}
\usepackage{amsmath}
\usepackage{amsfonts}
\usepackage{amssymb}
\usepackage{graphicx}
\usepackage{lmodern}
\usepackage{array}
\usepackage{multirow}
\AddThinSpaceBeforeFootnotes % à insérer si on utilise \usepackage[french]{babel}
\FrenchFootnotes % à insérer si on utilise \usepackage[french]{babel}



\linespread{1.2}

\usepackage{geometry}
\geometry{hmargin=2cm,vmargin=1.5cm}

\usepackage{soul} \usepackage{color} \newcommand{\hilight}[1]{\colorbox{yellow}{#1}}

\renewcommand{\thefootnote}{\roman{footnote}}

\title{Bilan moral du Festival Mayday}
\author{$6^e$ édition - 2024}
\date{Vendredi 31 mai 2024}

\makeindex

\begin{document}
\maketitle

\begin{abstract}
\textbf{A REFAIRE} \\
Le Mayday Festival est une manifestation culturelle. Mêlant musique et sensibilisation aux transitions socio-environnementales, ce festival s'articule autour d'objectifs simples et concrets :
\begin{itemize}
\item Ouverture de l’accès à la culture à toutes et tous,
\item Sensibilisation du plus grand nombre aux transitions environnementales et sociétales,
\item Être un laboratoire d’incubation, chef de fil en matière d’événementiel éco-responsable,
\item Mettre en avant des artistes locaux, émergents et étudiants,
\item Faire du campus non plus seulement un lieu de passage et d'étude, mais aussi un lieu de vie que les étudiants s'approprient,
\item \sloppy Favoriser l’ouverture du monde académique aux arts vivants et vers le territoire métropolitain.
\end{itemize}
Dans le but d’être ouvert à plus de public et de permettre le passage de davantage d’artistes, la 6e édition du festival se déroulait sur 2 jours. Cette évolution ambitieuse, mais cohérente avec les objectifs, fera l’objet d’une analyse dans ce rapport. Véritable vitrine et vecteur de l'engagement étudiant sur les campus bordelais, le Mayday est une expérience rassemblant de nombreux partenaires, acteurs du territoire, services publics et étudiants. Plusieurs périodes se sont succédé tout au long de la vie du projet ; ces séquences seront détaillées dans le rapport, avant les bilans et les conclusions que nous tirons de cette édition.
\end{abstract}

\tableofcontents

\newpage

\section*{Avant-propos}
%\addcontentsline{toc}{section}{Avant-propos}

Le Mayday Festival est une manifestation culturelle. Mêlant musique et sensibilisation aux transitions socio-environnementales, ce festival s'articule autour d'objectifs simples et concrets :
\begin{itemize}
\item Ouverture de l’accès à la culture à toutes et tous,
\item Sensibilisation du plus grand nombre aux transitions environnementales et sociétales,
\item Être un laboratoire d’incubation, chef de fil en matière d’événementiel éco-responsable,
\item Mettre en avant des artistes locaux, émergents et étudiants,
\item Faire du campus non plus seulement un lieu de passage et d'étude, mais aussi un lieu de vie que les étudiants s'approprient,
\item Favoriser l’ouverture du monde académique aux arts vivants et vers le territoire métropolitain.
\end{itemize}
Dans le but d’être ouvert à plus de public et de permettre le passage de davantage d’artistes, la 6e édition du festival se déroulait sur 2 jours. Cette évolution ambitieuse, mais cohérente avec les objectifs, fera l’objet d’une analyse dans ce rapport. Véritable vitrine et vecteur de l'engagement étudiant sur les campus bordelais, le Mayday est une expérience rassemblant de nombreux partenaires, acteurs du territoire, services publics et étudiants. Plusieurs périodes se sont succédé tout au long de la vie du projet ; ces séquences seront détaillées dans le rapport, avant les bilans et les conclusions que nous tirons de cette édition.

Ce bilan moral est l’occasion pour les organisateurs de revenir sur l’édition 2024 du Mayday. Commençant par son déroulement, afin d’établir une liste de ses points forts et de ses fragilités permettant de faire ressortir des axes d’amélioration pour les prochaines éditions, nous poursuivrons par une partie dédiée à l'équipe organisatrice, puis un retour sur les innovations mises en œuvre sur cette édition. Nous reviendrons sur le passage à 2 jours de festival et proposerons une analyse des points faibles et forts permettant de mieux appréhender les évolutions à mettre en place sur les prochaines éditions. Un retour d'expérience par séquence de l'événement (organisation, installation, exploitation, démontage) sera ensuite exposé. Ce bilan sera clôturé par une discussion et la formulation d'horizons pour le festival Mayday.

\section*{Contexte financier}
\paragraph*{État de la trésorerie suite au Mayday 2023 (mai - juillet 2023)}
La sixième édition du festival a été réalisée dans un contexte financier compliqué pour l’association. En effet, le bilan de l'édition 2023 a fait état d'une situation financière tendue, marquée notamment par un déficit de 36 733,11 €. Cette situation met en évidence un manque de maîtrise des dépenses, de contrôle par les organes compétents\footnote{Définit dans les statuts de l'association M-Tech, il s'agit du conseil d'administration} et une programmation budgétaire insuffisamment étayée, ne permettant pas d'anticiper certains écueils financiers (plusieurs subventions sollicitées non acquises). Par ailleurs, il apparaît que le maintien de certains engagements contractuels, malgré leur poids financier et les contraintes techniques croissantes, a contribué à la situation déficitaire\footnote{La direction sécurité sûreté l'université de Bordeaux et les personnes responsables de la sécurité du Mayday ayant alerté le directeur du festival sur les contraintes techniques, de sécurité et les frais supplémentaires inhérent au tir d'un feu d'artifice dans un délai permettant raisonnablement l'annulation partielle de la prestation, entrainant ainsi une facturation inférieure en application des conditions générales de vente}.

\paragraph*{Décision prises dans le périmètre trésorerie (septembre - octobre 2023)}
L'association a vu une part significative de ses liquidités consommées, le passif exigible\footnote{Définit plus haut à hauteur de 36 733,11 €, mais ne semble pas concordant avec les montants des échéanciers effectivement programmés et partiellement réalisés sur les exercices 2023 et 2024.} ne semblant pas excéder l'actif qui serait disponible mensuellement\footnote{d'après la programmation établie (événements, subventions et autres ressources de l'association) par les personnes responsables des comptes sur les mandats 2022-2023 et 2023-2024} une échéance mensuelle des montants dû a été négociée pour une période de 10 mois.

\paragraph*{Éléments communiqués durant la préparation du Mayday 2024 (décembre 2023 - avril 2024)}
En outre, des éléments complémentaires ont été communiqués aux organisateurs de l'édition 2024\footnote{Référence à la $6^e$ édition du Mayday, dont une partie des membres sont rédacteurs de ce document} au mois de décembre 2023 et au mois d'avril 2024. Force est de constater que ces éléments avaient portés à la connaissance des dirigeants de l'association\footnote{Les dirigeants de l'association M-Tech correspondent au bureau restreint défini dans les statuts} en amont de leur transmission aux équipes chargées du projet \og Mayday 2024 \fg{}. Ces délais, ayant couru dans des conditions qui restent à établir, ont déstabilisé la trésorerie de l'association et le contrôle exercé par les organes compétents, sans pour autant qu'un impact sur le projet en cours puisse être identifié, le budget de l'événement cité étant équilibré en recettes et en dépenses.

\paragraph*{Engagements et enjeux pour l'édition du Mayday 2024 (juillet - septembre 2023)}
Il convient donc d'établir que la situation de l'association au moment du lancement du projet \og Mayday 2024 \fg{} devrait être stabilisée par une gestion maîtrisée, contrôlée par les organes compétents et anticipant de potentiels aléas\footnote{Pouvant couvrir un large spectre : budgets d'événements non consolidés aux risques météorologique, contrôlables ou non.}.

\section*{Un projet unique}

Le Mayday Student Music Festival est un projet unique : il s'agit du plus grand festival gratuit, en France, porté par des étudiants venus de différentes formations de l’université de Bordeaux et de l’université Bordeaux Montaigne. Ces étudiants se réunissent et s’investissent bénévolement au sein du projet pendant une année entière. Cet engagement est fort, réalisé en parallèle de leurs études, parfois d’emplois étudiant, d’autres activités associatives ou représentatives. Il encourage une montée en compétences, permet d'acquérir de nouvelles connaissances et un savoir-être important. Ces étudiants vont développer leur compétence en gestion de projet (rétroplanning, mailing, reporting,… ), leur agenda est souvent bien rempli et il en est de même pour leur boîte mail. C’est aussi une expérience concrète de travail en équipe, où la communication et l'entraide sont primordiales. Enfin, c’est une introduction, parfois même une découverte, au monde professionnel durant laquelle les organisateurs vont travailler avec les établissements partenaires, un grand nombre d'entreprises, d’associations et de collectivités locales. L’engagement bénévole des étudiants et les soft-skills\footnote{ Aussi appelées compétences douces ou compétences comportementales, ce sont des aptitudes personnelles et relationnelles qui permettent de mieux interagir avec les autres et de s'adapter à différentes situations. Ils se distinguent des hard skills, qui sont les compétences techniques et les connaissances spécifiques à un métier.} développés lors de l’expérience Mayday sont reconnus et valorisés par les employeurs.

Le Mayday est un évènement unique. De par son lien avec les universités bordelaises, le CROUS les collectivités territoriales, premier festival étudiant de France, il fait rayonner l’Université et ses valeurs ; notamment la manière dont elle accompagne ses étudiants dans leur projet. Il s’agit d’un évènement fédérateur pour toute la communauté universitaire. Le Mayday reçoit le soutien de 11 services universitaires. 

\section*{Remerciements}

\newpage

\chapter{Déroulement de l'événement}

\section{Vendredi 17 mai}

Le festival a ouvert ses portes à 18h50. 20 minutes de retard sont dues à un imprévu de dernière minute (déclenchement inopiné de l'alarme incendie dans le bâtiment A21). Le DJ set de Ekograal a démarré comme prévu à 18h30, mais le public ne pouvant pas entrer sur site il s'est brièvement arrêté. Le set a ensuite redémarré lorsque le public a pu pénétrer sur le site (18h50). Un temps entre chaque concert avait été prévu en cas de besoin, mais sa durée n'était pas suffisante pour permettre d'annuler le retard dès le passage du premier artiste. Un décalage des sets a permis de diminuer ce retard durant la soirée. Fin des concerts à 00h10 au lieu de minuit initialement prévu.

L'évacuation du public s'est réalisée sans difficulté, et le site était vide à 0h25. Il n'y a pas eu de formation d'attroupement sur la voie publique grâce au concours des agents TBM et de bénévoles qui ont immédiatement conduits les festivaliers vers les transports en commun. L'interdiction de circulation sur l'esplanade de la bibliothèque sciences et techniques et des voies y menant a permis une gestion optimale des flux.

Les bars ont fonctionné de manière ininterrompue de 18h50 à 0h (vente de bière blonde, blanche et de limonade ; eau gratuite). Nous avons noté qu'une augmentation du débit du bar à eau pourrait être intéressante de manière à permettre un accès plus rapide à l'eau.

Sur le volet restauration, 7 food trucks proposant une offre 100\% végétarienne étaient présent. Suite à notre sollicitation, une partie des gérants nous ont fait un retour : il semblerait que de multiples facteurs aient menés à ce que peu de ventes soient réalisées (horaires du festival, météo, contexte d'inflation). De plus, un stand de pâtisserie végan fût tenu par Le Maraudeur : étudiant montant en renommée sur les réseaux sociaux, il s'engage au travers de pâtisseries végan, peu onéreuses et ne contenant par les allergènes les plus communs. \\

Plusieurs partenaires du festival étaient venus partager leurs actions à l'occasion de cette soirée : 
\begin{itemize}
\item la société Karos, qui gère une plateforme de co-voiturage - également présente sur les campus au travers de l'université de Bordeaux,
\item le CROUS Bordeaux-Aquitaine, opérateur de service public, qui est venu partager la tenue de la finale nationale du tremplin musical pulsations la semaine suivante (l'équipe du Mayday a d'ailleurs tenu la buvette à cette occasion !)
\end{itemize}

Au niveau sécurité/secours/prévention, des inspections visuelles, palpations et détections de métaux ont permis de prévenir les risques du festival. Il n'y a pas eu d'évacuations, et la protection civile a fait une seule prise en charge (crise d'épilepsie, patiente avec historique). Le placement de la safe zone, à l'écart de la foule, a permis qu'elle soit un lieu confidentiel. Nous notons cependant qu'une communication renforcée est nécessaire pour que le dispositif soit connu et plébiscité. 

Près de 200 bénévoles ont apporté leur concours lors de cette soirée. A l'issue d'une formation dispensée plus tôt dans le mois, ils ont occupé des postes variés. Leur arrivée sur site s'est faite au compte goute, sur tout la journée. Avant l'ouverture, un briefing général a eu lieu. Il a permis de faire le tour du site et des postes sur lesquels les bénévoles seront amenés à contribuer. Les retours sur leurs missions ont été particulièrement positifs, à l'exception de certains postes où l'activité menée pouvait être passive. En guise de remerciement, nous offrions 2 consommations/bénévoles ainsi que les repas. Cependant, et comme sur l'édition précédente, nous notons une franche difficulté à recruter des bénévoles qui se retrouvent alors à nous aider, parfois bien plus, que ce qui était initialement prévu.

Une gestion des déchets ambitieuse fût mise en œuvre cette année. Accompagné par des brigades de bénévoles, plusieurs types de déchets avaient étés identifiés. Ils ont pu être quantifiés et ces informations seront partagées dans le rapport d'activité de la $6^e$ édition. De plus, une enquête a été mené auprès d'un échantillon de festivaliers. Elle nous a permis de récolter des données très intéressantes pour le pilotage du festival (mode de venue, canal de communication, ...). Ces données seront également disponibles, après leur traitement, dans le rapport d'activité.

Cette soirée fût cependant humide, avec plusieurs courtes averses pendant les concerts. A ce jour nous avons mesuré un double impact : fréquentation en deçà de nos estimations (notamment vis-à-vis des retours sur les réseaux sociaux) et des ventes (ressources essentielles pour le festival) faibles.

\section{Samedi 18 mai}

Les portes du festival ont ouvert à 14h15 (15 minutes après l'horaire annoncé). En conséquence la table ronde prévue a été repoussée. L'organisation ayant prévu ce potentiel retard, la suite du programme n'a pas été affectée par ce retard.

Jusqu'à 18h, se sont tenus les temps forts : table ronde, labioratoire, friperie, jeux divers, spectacle vivant (en partenariat avec l'IUT Bordeaux Montaigne) et goûter solidaire. Nous avons eu le plaisir de recevoir une visite du Maire de Talence, il a notamment pu échanger l'association Ma famille extraordinaire, qui intervenait à l'occasion du goûter solidaire.

Un aléa météorologique est subi aux alentours de 18h20, durant jusqu'à 18h40. La réactivité des équipes et l'anticipation ont permit de limiter les avaries techniques. Sur le même modèle que la veille, les concerts suivant ont été légèrement repoussés mais le retard a été absorbé par le temps de battement réduit entre les sets. Nous notons cependant qu'un temps de battement plus important entre les sets pourrait être propice, marquant une séparation entre les artistes et permettant de profiter d'un temps plus calme avant d'être de nouveau emporté par la musique.

Présent en nombre plus élevée lors de cette soirée, le public est évacué à la fin des concerts et se disperse sans heurt. A nouveau, le concours des agents TBM a facilité la dispersion. Le site est vide à 1h30.

Durant l'après-midi, les buvettes ont fonctionné à 50\% de leur possibilité. En effet, la demande étant moins élevée nous avons préféré que les bénévoles profitent de temps libre. 

Les même food-trucks que la veille étaient présents, les ventes furent légèrement supérieures. 
Sur la soirée, le CROUS Bordeaux-Aquitaine fût à nouveau présent.

Sur le périmètre sécurité/secours/prévention, un dispositif plus conséquent que la veille a été déployé. Aucune victime ne furent à déplorer, et la safe zone connut une fréquentation légèrement supérieure. 

Dans la continuité du dispositif de gestion des déchets mis en place le vendredi, un dispositif amélioré suivant les habitudes et besoins observées a été mis en place.

\section{Affluence}
\begin{center}
\begin{tabular}{|c|c|c|}
Vendredi soir & Samedi après-midi & Samedi soir \\
\hline
4000 & 400 & 8000\\
\end{tabular}
\end{center}

Sur l'ensemble des deux jours, nous avons comptabilisé 17 000 entrées. Un bilan en légère baisse par rapport à l'édition précédente, mais justifié par la météo changeante.

%\section{Contexte}
%Les années marquées par la pandémie "Covid-19" ont eu un fort impact sur la transmission des compétences, connaissances et réseaux des membres d'associations. Le Festival Mayday était porté depuis 2016 par le même noyau dûr, puis repris par une équipe majoritairement novice pour la 5e édition (2023). L'objectif de cette année était notamment de capitaliser sur les connaissances et réseaux restaurés sur les années précédentes. 
%
%\section{Méthodologie}
%\subsection{Recrutement}
%De nombreux membres de l'organisation ayant fini leurs études, l'équipe d'organisation s'est largement renouvelée en fin 2023.
%Le recrutement s'est fait via plusieurs méthodes : 
%\begin{itemize}
%\item Réseau proche
%\item Annonce en ligne
%\end{itemize}
%
%\subsubsection{Réseau proche}
%Au travers d'expériences passées, les membres de l'organisation avaient pu interagir avec d'autres étudiants. De nombreux profils avaient étés pré-identifiés, cependant l'ampleur du projet, l'investissement nécessaires et les attendus sur les postes ouverts peuvent être des facteurs limitant.
%
%\subsection{Pôles}
%\subsubsection{Enjeux}
%L'organisation du festival se construit selon plusieurs enjeux : 
%\begin{itemize}
%\item Coordination du projet,
%\item Sécurité de l'événement,
%\item Déclaration administratives,
%\item Programmation événementielle,
%\item Matériel,
%\item Électricité,
%\item Sensibilisation,
%\item Bilan carbone,
%\item Communication,
%\item Partenariats,
%\item Gestion des bénévoles,
%\item Gestion financière,
%\item Restauration sur site (nourriture et boisson). \\
%\end{itemize}
%
%Une organisation en pôles a donc été mise en place, avec 2 type de pôle : 
%\begin{itemize}
%\item Orienté
%\item Transverse/Support
%\end{itemize}
%
%\subsubsection{Version finale}
%Plusieurs essais ont étés nécessaires, avec de nombreux ajustements. Nous avons obtenu cette liste (dans l'ordre alphabétique) : 
%\begin{itemize}
%\item Administratif
%\item Artistes
%\item Bar
%\item Bénévoles
%\item Communication
%\item Innovation écologique et sociale
%\item Logistique
%\item Partenariats
%\item Presse
%\item Restauration
%\item Sécurité technique prévention
%\item Transverse
%\item Trésorerie
%\end{itemize}
%
%\subsection{Interactions}
%Pour organiser cette 6e édition, de nombreux échanges entre les différentes entités crées devait se faire. Nous pouvons les catégoriser en plusieurs types, à voir ci-dessous.
%
%\subsubsection{Administratif}
%Dans le cadre de déclarations (sécurité notamment), de nombreux échanges étaient nécessaires pour que des informations à jour soient communiqués. Ils comprenaient plusieurs volets :
%\begin{itemize}
%\item Implantation (plans, simulations),
%\item Besoins spécifiques,
%\item Risques spécifiques,
%\item Utilisations particulières de matériel (non conforme à l'usage préconisé),
%\item Installation de CTS (Chapiteaux, tentes, structures),
%\item Plans alternatifs (météo)
%\item Suivi des dates limites de dépôts, échanges avec les autorités, suivi des nouvelles réglementations.
%\end{itemize}
%
%\subsubsection{Communication et partenariats}
%Dans le but de mettre en avant le travail fait bénévolement par ses membres, plusieurs personnes avaient pour mission de communiquer sur :
%\begin{itemize}
%\item l'événement,
%\item les engagements,
%\item la gestion de projet,
%\item les manifestations annexes (participations à des concours, tenue de buvettes, ...)
%\item ...
%\end{itemize}
%
%Pour cela, des échanges réguliers et la présence de l'équipe de communication en réunion hebdomadaire permettaient une remontée conforme de nos avancées et de la direction prise par le projet auprès du grand public.
%
%Sur le plan partenarial (public professionnel, politiques, administrations), la mise en avant des enjeux de la manifestation, ainsi que les engagements humains derrière la façade festive et engagée. Elle se faisait notamment au travers d'une couverture photo/vidéo de nos interventions (séminaires, demandes de subventions, réunions).
%
%
%\subsubsection{Montage de projet}
%De manière plus générale, l'ensemble du montage du projet nécessitait des échanges globaux. Des point hebdomadaires étaient organisés, permettant que l'ensemble de l'équipe ait le même niveau d'information.


\chapter{Équipe organisatrice}

Une équipe motivée et dynamique est l'essence du festival. 

Cette année, l'équipe du Mayday était composée par :
\begin{center}
\begin{tabular}{| m{3cm} | m{9cm} |}
\hline
Pôle & Prénom \\ 
 \hline\hline
Transverse & \textbf{Louis Delignac}\\ 
\hline
\multirow{2}{5em}{Sécurité technique prévention} & \textbf{Sacha Duperret} \\ 
& Antoine Philippeau, Ebène Bilé, Maël Avadian, Flavio Arella\\  
 \hline
 \multirow{2}{5em}{Innovation écologique et sociale} & \textbf{Baptiste Royau} \\ 
& Katarina Montenier, Manon Bénard, Eléa Peynaud, Leelou Cateigbou\\
 \hline
  \multirow{2}{5em}{Trésorerie} & \textbf{Aurélien Gauthier} \\ 
& Victor Lamare\\
 \hline
 \multirow{2}{5em}{ Artistes} & \textbf{Leïla Karim} \\
& Lucile Guerin\\
 \hline
\multirow{2}{5em}{Bénévoles} & \textbf{Nabil Es-Selymy} \\ 
& Anessa Diallo\\
 \hline
\multirow{2}{7em}{Communication} & \textbf{Alice Conti} \\ 
& Salma Rabion, Nassim Sellal, Julie Fourneyron\\
 \hline
 \multirow{1}{5em}{Logistique} & \textbf{Arthur Vienot} \\ 
 \hline
 \multirow{1}{6em}{Restauration} & \textbf{Lucas Brouet} \\ 
 \hline
\end{tabular}
\end{center}

\chapter{Innovations de l'édition 2024}

\section{Analyse passage sur 2 jours}

\chapter{Retour d'expérience}

\section{Phase d'organisation (août 2023 - mai 2024}

\section{Phase d'installation (13 - 17 mai 2024}

\section{Phase d'exploitation (17 - 18 mai 2024}

\section{Phase de démontage (19 mai - juin 2024}

%\part{Installation et mise en place (10 mai - 17 mai 2024)}
%\chapter{Livraisons}
%Du vendredi 10 mai au vendredi 17 mai, le festival fut mis en place. 
%\section{Méthodologie}
%
%
%\section{Opérationnel}
%\subsection{Sécurité}
%Plusieurs enjeux de sécurité coexistaient durant les temps d'installation et de mise en place du festival : 
%\begin{itemize}
%\item Sécurisation des livraisons,
%\item Protection des usagers classiques du campus,
%\item Stockage du matériel (nuit et intempéries).
%\end{itemize} 
%
%
%Le festival se déroulant sur le campus universitaire de sciences durant une période d'activité, l'organisation du festival devait s'adapter : 
%\begin{itemize}
%\item Niveau de bruit faible pendant les horaires d'enseignements ainsi que la nuit (voisinage),
%\item Maintenir les accès et signaler les déviations, notamment PMR et vélo, entre les différents bâtiments et pour traverser le campus,
%\item Partager les espaces intérieurs avec les utilisateurs habituels (services, enseignants, associations étudiantes, ...).
%\end{itemize}
%
%Pour cela, un étalement de l'installation a été mis en place. Habituellement commençant au milieu de la semaine, les livraisons ont commencées dès le vendredi précédent. Cela permettait particulièrement de mieux gérer les flux d'usagers du campus.
%
%Par ailleurs, le soutien de l'université de Bordeaux au travers de son service scolarité du collège sciences et technologies était ici primordial. Il a permis le déplacement de nombreux enseignements à distance des espaces "à risques de nuisances" vers des lieux plus calme, à proximité.
%
%Un arrêté du président de l'UB prévoyait, dès le début de la semaine, la fermeture de plusieurs axes de circulation (piéton, vélo et VL). Cela permettait de :
%\begin{itemize}
%\item Mettre en place un barriérage de sécurité autour du site (installation progressive suivant l'installation),
%\item Ouvrir des voies d'accès sécurisée depuis et vers les bâtiments ainsi que les sorties du campus,
%\item Faciliter la sécurisation des livraisons, dont les manœuvres pouvaient avoir lieu dans un environnement contrôlé,
%\item Concourir à la réussite du travail des agents de sécurité en mission de gardiennage de nuit.
%\end{itemize}
%
%
%
%\section{Bilan}
%
%\part{Opérationnel (17 mai - 18 mai 2024)}
%\section{Sécurité}
%
%
%\part{Désinstallation et rangement (19 mai - 7 juin 2024)}
%
%\part{Bilan moral}
%\chapter{Partenaires}
%
%\part{Bilan financier}
%\chapter{Analyse financière - passage à 2 jours}
%
%\subsection{subsection}
%\subsubsection{subsubsection}
%\paragraph{paragraph}
%\subparagraph{subparagraph}

\end{document}