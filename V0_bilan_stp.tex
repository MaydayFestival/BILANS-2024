\documentclass[12pt,a4paper]{report}
\usepackage[utf8]{inputenc}
\usepackage[french]{babel}
\usepackage[hidelinks]{hyperref}
\usepackage{microtype}
\usepackage[T1]{fontenc}
\usepackage{helvet}
\renewcommand{\familydefault}{\sfdefault}
\usepackage{amsmath}
\usepackage{amsfonts}
\usepackage{amssymb}
\usepackage{graphicx}
\usepackage{lmodern}
\usepackage{array}
\usepackage{multirow}
\AddThinSpaceBeforeFootnotes % à insérer si on utilise \usepackage[french]{babel}
\FrenchFootnotes % à insérer si on utilise \usepackage[french]{babel}



\linespread{1.2}

\usepackage{geometry}
\geometry{hmargin=2cm,vmargin=1.5cm}

\usepackage{soul} \usepackage{color} \newcommand{\hilight}[1]{\colorbox{yellow}{#1}}

\renewcommand{\thefootnote}{\roman{footnote}}

\title{Bilan moral du Festival Mayday \\ $6^e$ édition - 2024}
\author{Équipe Mayday 2024}
\date{Vendredi 31 mai 2024}

\makeindex

\begin{document}

\maketitle

\begin{abstract}
Le bilan moral de la 6ème édition du festival Mayday 2024 met en lumière les succès et les défis rencontrés lors de la réalisation de cet événement de deux jours. Malgré un contexte financier difficile hérité de l’édition précédente, l’équipe a réussi à équilibrer le budget grâce à une gestion rigoureuse et à l’anticipation des risques. L’expansion du festival sur deux jours a permis d’accueillir plus d’artistes et de proposer un programme plus riche.\\

L’engagement de plus de 200 bénévoles a été essentiel à la réussite du festival, contribuant à l’organisation, à la sécurité, à la gestion des déchets et à l’accueil du public. Des améliorations ont été apportées, notamment en matière de restauration avec une offre 100\% végétarienne et une cuisine maison pour les équipes. Le festival a également mis en place une \og Safe zone \fg{} pour garantir la sécurité et le bien-être des festivaliers.\\

Malgré une fréquentation légèrement inférieure aux attentes en raison de la météo, le bilan financier de cette édition est équilibré. L’augmentation du nombre de partenaires et des ventes additionnelles ont permis de compenser les coûts supplémentaires liés à l’extension du festival sur deux jours.\\

Cependant, des défis subsistent, notamment en ce qui concerne la répartition des tâches au sein de l’équipe organisatrice et la gestion de la phase de démontage. Le bilan souligne également l’importance d’une communication renforcée pour mieux faire connaître les dispositifs de sécurité et de prévention mis en place lors du festival. Dans l’ensemble, le bilan moral de la $6^e$ édition du festival Mayday est positif, mettant en évidence la capacité de la nouvelle équipe à relever les défis et à innover pour offrir un événement culturel de qualité, accessible à tous et respectueux de l’environnement.
\end{abstract}

\tableofcontents

\newpage

\section*{Avant-propos}
\addcontentsline{toc}{subsection}{Avant-propos}

Le Mayday Festival est une manifestation culturelle. Mêlant musique et sensibilisation aux transitions socio-environnementales, ce festival s'articule autour d'objectifs durables et ambitieux :
\begin{itemize}
\item Ouverture de l’accès à la culture à toutes et tous,
\item Sensibilisation du plus grand nombre aux transitions environnementales et sociétales,
\item Être un laboratoire d’incubation, chef de fil en matière d’événementiel éco-responsable,
\item Mettre en avant des artistes locaux, émergents et étudiants,
\item Faire du campus non plus seulement un lieu de passage et d'étude, mais aussi un lieu de vie que les étudiants s'approprient,
\item Favoriser l’ouverture du monde académique aux arts vivants et vers le territoire métropolitain.\\
\end{itemize}

Dans le but d’être ouvert à plus de public et de permettre le passage de davantage d’artistes, la 6e édition du festival se déroulait sur 2 jours. Cette évolution ambitieuse, mais cohérente avec les objectifs, fera l’objet d’une analyse dans ce rapport. Véritable vitrine et vecteur de l'engagement étudiant sur les campus bordelais, le Mayday est une expérience rassemblant de nombreux partenaires, acteurs du territoire, services publics et étudiants. Plusieurs périodes se sont succédé tout au long de la vie du projet ; ces séquences seront détaillées dans le rapport, avant les bilans et les conclusions que nous tirons de cette édition.\\

\textbf{AJOUTER LE CONTEXTE DANS LEQUEL S'INSCRIT LE FESTIVAL}\\

Ce bilan moral est l’occasion pour les organisateurs de revenir sur l’édition 2024 du Mayday. Commençant par son déroulement, afin d’établir une liste de ses points forts et de ses fragilités permettant de faire ressortir des axes d’amélioration pour les prochaines éditions, nous poursuivrons par une partie dédiée à l'équipe organisatrice, puis un retour sur les innovations mises en œuvre sur cette édition. Nous reviendrons sur le passage à 2 jours de festival et proposerons une analyse des points faibles et forts permettant de mieux appréhender les évolutions à mettre en place sur les prochaines éditions. Un retour d'expérience par séquence de l'événement (organisation, installation, exploitation, démontage) sera ensuite exposé. Ce bilan sera clôturé par une discussion et la formulation d'horizons pour le festival Mayday.

\subsection*{Un projet unique}
\addcontentsline{toc}{subsection}{Un projet unique}

Le Mayday Student Music Festival est un projet unique : il s'agit du plus grand festival gratuit en France et porté par des étudiants, tous issus de différentes formations. Ces bénévoles se réunissent et s'investissent au sein de ce projet pendant une année (de août 2023 pour la préparation de l'édition 2024, à juillet 2024 pour la rédaction du bilan de l’événement). Cet engagement apparaît très fort au regard des autres activités dans lesquelles ces personnes sont impliquées : étude, emploi étudiant, autres mandats associatifs ou représentatifs.\\

\sloppy Le temps investi est cependant sanctionné par l'acquisition de compétences, connaissances et savoir-être importants. À titre d'exemple, ces étudiants développent des compétences de gestion de projet (rétroplanning, mailing, suivi de tâches, …), organisationnelles (nombreuses réunions, suivi de relations avec des partenaires et prestataires) et relationnelles (communication par mail, téléphonique, réunions). C’est aussi une expérience concrète de travail en équipe, où la communication et l'entraide sont primordiales. Enfin, c’est une introduction, parfois même une découverte, au monde professionnel durant laquelle les organisateurs vont travailler avec les établissements partenaires, un grand nombre d'entreprises, d’associations et de collectivités locales.\\

L’engagement bénévole des étudiants et les soft-skills\footnote{Ce sont des aptitudes personnelles et relationnelles qui permettent de mieux interagir avec les autres et de s'adapter à différentes situations. Ils se distinguent des hard skills, qui sont les compétences techniques et les connaissances spécifiques à un métier.} développés lors de l’expérience Mayday sont reconnus et valorisés par les employeurs.\\

Le Mayday est un évènement unique. De par son lien avec les universités bordelaises, le CROUS, les collectivités territoriales. Il fait rayonner l'ensemble de ses partenaires au travers de ses valeurs. Il s’agit d’un évènement fédérateur pour toute la communauté universitaire et métropolitaine.

\subsection*{Contexte financier}
\addcontentsline{toc}{subsection}{Contexte financier}

\paragraph*{État de la trésorerie suite au Mayday 2023 (mai - juillet 2023)}
La sixième édition du festival a été réalisée dans un contexte financier difficile pour l’association. En effet, le bilan de l'édition 2023 a fait état d'une gestion financière défaillante, marquée notamment par un déficit de 36 733,11 €. Cette situation met en évidence un manque de maîtrise des dépenses, de contrôle par les organes compétents\footnote{Définit dans les statuts de l'association M-Tech, il s'agit du conseil d'administration et du bureau} et une programmation budgétaire insuffisamment étayée, ne permettant pas d'anticiper certains écueils financiers (plusieurs subventions sollicitées non acquises). Par ailleurs, il apparaît que le maintien de certains engagements contractuels, malgré leur poids financier et les contraintes techniques croissantes, a contribué à la situation déficitaire\footnote{\sloppy La direction sécurité sûreté l'université de Bordeaux et les responsables de la sécurité du Mayday ayant alerté le directeur du festival sur les contraintes techniques, de sécurité et les frais supplémentaires inhérent au tir d'un feu d'artifice dans un délai permettant raisonnablement l'annulation partielle de la prestation}.

\paragraph*{Décision prises dans le périmètre trésorerie (septembre - octobre 2023)}
L'association a vu ses liquidités significativement consommées, le passif exigible ne semblant pas excéder l'actif qui serait disponible mensuellement, d'après la programmation établie par les personnes responsables des comptes, une échéance mensuelle des montants dûs a été négociée pour une période de 10 mois.

\paragraph*{Engagements et enjeux pour l'édition du Mayday 2024 (juillet - septembre 2023)}
Il convient donc d'établir que la situation de l'association au moment du lancement du projet “Mayday 2024” devait être stabilisée par une gestion maîtrisée, contrôlée par les organes compétents et anticipant de potentiels aléas (risques météorologiques, de sécurité ou sanitaires).

\newpage
\section*{Remerciements}
\addcontentsline{toc}{section}{Remerciements}

L’équipe du Mayday tient à remercier chaleureusement toutes les personnes qui ont contribué à la réussite de cette sixième édition. Nous sommes particulièrement reconnaissants envers :
\begin{itemize}
\item Les bénévoles, dont l’engagement et le dévouement ont été essentiels à la réalisation de cet événement. Leur énergie et leur enthousiasme ont été une source d’inspiration pour toute l’équipe.
\item Les partenaires, dont le soutien financier et matériel a permis de concrétiser nos ambitions et d’offrir un festival de qualité, gratuit et accessible à tous.
\item Les services de l’université de Bordeaux, dont l’accompagnement et la collaboration ont été précieux pour assurer le bon déroulement du festival sur le campus.
\item Le CROUS Bordeaux-Aquitaine, pour son accompagnement quotidien, pour le partenariat annuel noué avec le Mayday, pour son aide précieuse dans la mise en place de la restauration et pour sa participation active à l’événement.
\item Les collectivités territoriales, qui ont soutenu notre initiative et ont permis de renforcer le lien entre l’université et la communauté locale.
\item Les artistes, qui ont partagé leur talent et leur passion avec le public, contribuant ainsi à la richesse et à la diversité de la programmation.
\item Les associations, qui ont enrichi le festival par leurs actions et leurs messages de sensibilisation.
\item Les food trucks, qui ont proposé une offre alimentaire variée et de qualité, en accord avec nos valeurs écologiques.
\item Le public, qui a répondu présent malgré la météo capricieuse et qui a contribué à l’ambiance festive et conviviale du festival.
\end{itemize}
Nous tenons également à remercier tous ceux qui, de près ou de loin, ont apporté leur pierre à l’édifice et ont fait de cette sixième édition un moment inoubliable. Nous espérons vous retrouver encore plus nombreux l’année prochaine pour célébrer la musique et l’engagement étudiant !

\chapter{Déroulement de l'événement}

\section{Vendredi 17 mai}

Le festival a ouvert ses portes à 18h50. 20 minutes de retard sont dues à un imprévu de dernière minute (déclenchement inopiné de l'alarme incendie dans le bâtiment A21). Le DJ set de Ekograal a démarré comme prévu à 18h30, mais le public ne pouvant pas entrer sur site il s'est brièvement arrêté. Le set a ensuite redémarré lorsque le public a pu pénétrer sur le site (18h50). Un temps entre chaque concert avait été prévu en cas de besoin, mais sa durée n'était pas suffisante pour permettre d'annuler le retard dès le passage du premier artiste. Un décalage des sets a permis de diminuer ce retard durant la soirée. Fin des concerts à 00h10 au lieu de minuit initialement prévu.

L'évacuation du public s'est réalisée sans difficulté, et le site était vide à 0h25. Il n'y a pas eu de formation d'attroupement sur la voie publique grâce au concours des agents TBM et de bénévoles qui ont immédiatement conduits les festivaliers vers les transports en commun. L'interdiction de circulation sur l'esplanade de la bibliothèque sciences et techniques et des voies y menant a permis une gestion optimale des flux.

Les bars ont fonctionné de manière ininterrompue de 18h50 à 0h (vente de bière blonde, blanche et de limonade ; eau gratuite). Nous avons noté qu'une augmentation du débit du bar à eau pourrait être intéressante de manière à permettre un accès plus rapide à l'eau.

Sur le volet restauration, 7 food trucks proposant une offre 100\% végétarienne étaient présent. Suite à notre sollicitation, une partie des gérants nous ont fait un retour : il semblerait que de multiples facteurs aient menés à ce que peu de ventes soient réalisées (horaires du festival, météo, contexte d'inflation). De plus, un stand de pâtisserie végan fût tenu par Le Maraudeur : étudiant montant en renommée sur les réseaux sociaux, il s'engage au travers de pâtisseries végan, peu onéreuses et ne contenant par les allergènes les plus communs. \\

Plusieurs partenaires du festival étaient venus partager leurs actions à l'occasion de cette soirée : 
\begin{itemize}
\item la société Karos, qui gère une plateforme de co-voiturage - également présente sur les campus au travers de l'université de Bordeaux,
\item le CROUS Bordeaux-Aquitaine, opérateur de service public, qui est venu partager la tenue de la finale nationale du tremplin musical pulsations la semaine suivante (l'équipe du Mayday a d'ailleurs tenu la buvette à cette occasion !)
\end{itemize}

Au niveau sécurité/secours/prévention, des inspections visuelles, palpations et détections de métaux ont permis de prévenir les risques du festival. Il n'y a pas eu d'évacuations, et la protection civile a fait une seule prise en charge (crise d'épilepsie, patiente avec historique). Le placement de la safe zone, à l'écart de la foule, a permis qu'elle soit un lieu confidentiel. Nous notons cependant qu'une communication renforcée est nécessaire pour que le dispositif soit connu et plébiscité. 

Près de 200 bénévoles ont apporté leur concours lors de cette soirée. A l'issue d'une formation dispensée plus tôt dans le mois, ils ont occupé des postes variés. Leur arrivée sur site s'est faite au compte goute, sur tout la journée. Avant l'ouverture, un briefing général a eu lieu. Il a permis de faire le tour du site et des postes sur lesquels les bénévoles seront amenés à contribuer. Les retours sur leurs missions ont été particulièrement positifs, à l'exception de certains postes où l'activité menée pouvait être passive. En guise de remerciement, nous offrions 2 consommations/bénévoles ainsi que les repas. Cependant, et comme sur l'édition précédente, nous notons une franche difficulté à recruter des bénévoles qui se retrouvent alors à nous aider, parfois bien plus, que ce qui était initialement prévu.

Une gestion des déchets ambitieuse fût mise en œuvre cette année. Accompagné par des brigades de bénévoles, plusieurs types de déchets avaient étés identifiés. Ils ont pu être quantifiés et ces informations seront partagées dans le rapport d'activité de la $6^e$ édition. De plus, une enquête a été mené auprès d'un échantillon de festivaliers. Elle nous a permis de récolter des données très intéressantes pour le pilotage du festival (mode de venue, canal de communication, ...). Ces données seront également disponibles, après leur traitement, dans le rapport d'activité.

Cette soirée fût cependant humide, avec plusieurs courtes averses pendant les concerts. A ce jour nous avons mesuré un double impact : fréquentation en deçà de nos estimations (notamment vis-à-vis des retours sur les réseaux sociaux) et des ventes (ressources essentielles pour le festival) faibles.

\section{Samedi 18 mai}

Les portes du festival ont ouvert à 14h15 (15 minutes après l'horaire annoncé). En conséquence la table ronde prévue a été repoussée. L'organisation ayant prévu ce potentiel retard, la suite du programme n'a pas été affectée par ce retard.

Jusqu'à 18h, se sont tenus les temps forts : table ronde, labioratoire, friperie, jeux divers, spectacle vivant (en partenariat avec l'IUT Bordeaux Montaigne) et goûter solidaire. Nous avons eu le plaisir de recevoir une visite du Maire de Talence, il a notamment pu échanger l'association Ma famille extraordinaire, qui intervenait à l'occasion du goûter solidaire.

Un aléa météorologique est subi aux alentours de 18h20, durant jusqu'à 18h40. La réactivité des équipes et l'anticipation ont permit de limiter les avaries techniques. Sur le même modèle que la veille, les concerts suivant ont été légèrement repoussés mais le retard a été absorbé par le temps de battement réduit entre les sets. Nous notons cependant qu'un temps de battement plus important entre les sets pourrait être propice, marquant une séparation entre les artistes et permettant de profiter d'un temps plus calme avant d'être de nouveau emporté par la musique.

Présent en nombre plus élevée lors de cette soirée, le public est évacué à la fin des concerts et se disperse sans heurt. A nouveau, le concours des agents TBM a facilité la dispersion. Le site est vide à 1h30.

Durant l'après-midi, les buvettes ont fonctionné à 50\% de leur possibilité. En effet, la demande étant moins élevée nous avons préféré que les bénévoles profitent de temps libre. 

Les même food-trucks que la veille étaient présents, les ventes furent légèrement supérieures. 
Sur la soirée, le CROUS Bordeaux-Aquitaine fût à nouveau présent.

Sur le périmètre sécurité/secours/prévention, un dispositif plus conséquent que la veille a été déployé. Aucune victime ne furent à déplorer, et la safe zone connut une fréquentation légèrement supérieure. 

Dans la continuité du dispositif de gestion des déchets mis en place le vendredi, un dispositif amélioré suivant les habitudes et besoins observées a été mis en place.

\section{Affluence}
\begin{center}
\begin{tabular}{|c|c|c|}
Vendredi soir & Samedi après-midi & Samedi soir \\
\hline
4000 & 400 & 8000\\
\end{tabular}
\end{center}

Sur l'ensemble des deux jours, nous avons comptabilisé 17 000 entrées. Un bilan en légère baisse par rapport à l'édition précédente, mais justifié par la météo changeante.

%\section{Contexte}
%Les années marquées par la pandémie "Covid-19" ont eu un fort impact sur la transmission des compétences, connaissances et réseaux des membres d'associations. Le Festival Mayday était porté depuis 2016 par le même noyau dûr, puis repris par une équipe majoritairement novice pour la 5e édition (2023). L'objectif de cette année était notamment de capitaliser sur les connaissances et réseaux restaurés sur les années précédentes. 
%
%\section{Méthodologie}
%\subsection{Recrutement}
%De nombreux membres de l'organisation ayant fini leurs études, l'équipe d'organisation s'est largement renouvelée en fin 2023.
%Le recrutement s'est fait via plusieurs méthodes : 
%\begin{itemize}
%\item Réseau proche
%\item Annonce en ligne
%\end{itemize}
%
%\subsubsection{Réseau proche}
%Au travers d'expériences passées, les membres de l'organisation avaient pu interagir avec d'autres étudiants. De nombreux profils avaient étés pré-identifiés, cependant l'ampleur du projet, l'investissement nécessaires et les attendus sur les postes ouverts peuvent être des facteurs limitant.
%
%\subsection{Pôles}
%\subsubsection{Enjeux}
%L'organisation du festival se construit selon plusieurs enjeux : 
%\begin{itemize}
%\item Coordination du projet,
%\item Sécurité de l'événement,
%\item Déclaration administratives,
%\item Programmation événementielle,
%\item Matériel,
%\item Électricité,
%\item Sensibilisation,
%\item Bilan carbone,
%\item Communication,
%\item Partenariats,
%\item Gestion des bénévoles,
%\item Gestion financière,
%\item Restauration sur site (nourriture et boisson). \\
%\end{itemize}
%
%Une organisation en pôles a donc été mise en place, avec 2 type de pôle : 
%\begin{itemize}
%\item Orienté
%\item Transverse/Support
%\end{itemize}
%
%\subsubsection{Version finale}
%Plusieurs essais ont étés nécessaires, avec de nombreux ajustements. Nous avons obtenu cette liste (dans l'ordre alphabétique) : 
%\begin{itemize}
%\item Administratif
%\item Artistes
%\item Bar
%\item Bénévoles
%\item Communication
%\item Innovation écologique et sociale
%\item Logistique
%\item Partenariats
%\item Presse
%\item Restauration
%\item Sécurité technique prévention
%\item Transverse
%\item Trésorerie
%\end{itemize}
%
%\subsection{Interactions}
%Pour organiser cette 6e édition, de nombreux échanges entre les différentes entités crées devait se faire. Nous pouvons les catégoriser en plusieurs types, à voir ci-dessous.
%
%\subsubsection{Administratif}
%Dans le cadre de déclarations (sécurité notamment), de nombreux échanges étaient nécessaires pour que des informations à jour soient communiqués. Ils comprenaient plusieurs volets :
%\begin{itemize}
%\item Implantation (plans, simulations),
%\item Besoins spécifiques,
%\item Risques spécifiques,
%\item Utilisations particulières de matériel (non conforme à l'usage préconisé),
%\item Installation de CTS (Chapiteaux, tentes, structures),
%\item Plans alternatifs (météo)
%\item Suivi des dates limites de dépôts, échanges avec les autorités, suivi des nouvelles réglementations.
%\end{itemize}
%
%\subsubsection{Communication et partenariats}
%Dans le but de mettre en avant le travail fait bénévolement par ses membres, plusieurs personnes avaient pour mission de communiquer sur :
%\begin{itemize}
%\item l'événement,
%\item les engagements,
%\item la gestion de projet,
%\item les manifestations annexes (participations à des concours, tenue de buvettes, ...)
%\item ...
%\end{itemize}
%
%Pour cela, des échanges réguliers et la présence de l'équipe de communication en réunion hebdomadaire permettaient une remontée conforme de nos avancées et de la direction prise par le projet auprès du grand public.
%
%Sur le plan partenarial (public professionnel, politiques, administrations), la mise en avant des enjeux de la manifestation, ainsi que les engagements humains derrière la façade festive et engagée. Elle se faisait notamment au travers d'une couverture photo/vidéo de nos interventions (séminaires, demandes de subventions, réunions).
%
%
%\subsubsection{Montage de projet}
%De manière plus générale, l'ensemble du montage du projet nécessitait des échanges globaux. Des point hebdomadaires étaient organisés, permettant que l'ensemble de l'équipe ait le même niveau d'information.


\chapter{Équipe organisatrice}

Une équipe motivée et dynamique est l'essence du festival. 

L’équipe organisatrice est composée de 23 jeunes étudiants, venant de formations différentes. Ils ont fait le choix de s’investir bénévolement pour ce projet ambitieux et des valeurs communes.

Il s'agit de la première année où l'équipe chargée de l'organisation est aussi nombreuse. Une organisation répartissant les missions par périmètre thématique a été mise en œuvre. A posteriori, il convient de faire évoluer l'organisation pour obtenir une répartition des tâches plus claire, faciliter le suivi et le management de profils néophytes. 

Plusieurs points forts sont à souligner : 
\begin{itemize}
\item engagement fort de l'équipe,
\item motivation et dynamique.
\end{itemize}

Cependant des points faibles le sont aussi : 
\begin{itemize}
\item ambiance en réunion durant une partie de l'année,
\item difficulté à mettre en place une attitude adaptée en fonction des temps\footnote{une attitude de travail en réunion et une attitude pouvant être plus amicale en dehors}.
\end{itemize}

Sans qu'ils rentrent nécessairement dans l'une ou l'autre de ces catégories, les points suivant ont aussi été identifiés :
\begin{itemize}
\item valeurs et ambitions du projet : projet inspirant, stimulant, partage et transmission,
\item typologie d'événement festive, culturelle, taille de l'événement,
\item engagement : 1 an, dépassement de soir, professionnalisation (découverte de vocation), prise de responsabilité, compétences à acquérir, expérience unique, pas ou peu de passation avec l'équipe précédente,
\item impacte de l'événement : sentiment d'utilité, sensibilisation, éducation populaire, médiation,
\item équipe : esprit d'équipe, sentiment d'appartenance, équipe motivée et volontaire.
\end{itemize}

Cette année, l'équipe du Mayday était composée par :
\begin{center}
\begin{tabular}{| m{3cm} | m{10cm} |}
\hline
Pôle & Prénom \\ 
 \hline\hline
Transverse & \textbf{Louis Delignac}\\ 
\hline
\multirow{2}{5em}{Sécurité technique prévention} & \textbf{Sacha Duperret} \\ 
& Antoine Philippeau, Ebène Bilé, Maël Avadian, Flavio Arella\\ 
 \hline
 \multirow{2}{5em}{Innovation écologique et sociale} & \textbf{Baptiste Royau} \\ 
& Katarina Montenier, Manon Bénard, Eléa Peynaud, Leelou Cateigbou\\
 \hline
 \multirow{2}{5em}{Trésorerie} & \textbf{Aurélien Gauthier} \\ 
& Victor Lamare\\
 \hline
 \multirow{2}{5em}{Artistes} & \textbf{Leïla Karim} \\
& Lucile Guerin\\
 \hline
\multirow{2}{5em}{Bénévoles} & \textbf{Nabil Es-Selymy} \\ 
& Anessa Diallo\\
 \hline
\multirow{2}{7em}{Communication} & \textbf{Alice Conti} \\ 
& Salma Rabion, Nassim Sellal, Julie Fourneyron\\
 \hline
 \multirow{1}{5em}{Logistique} & \textbf{Arthur Vienot} \\ 
 \hline
 \multirow{1}{5em}{Restauration} & \textbf{Lucas Brouet} \\ 
 \hline
\end{tabular}
\end{center}

\textit{Les responsables de chaque pôle sont identifiés en gras.}

\chapter{Innovations de l'édition 2024}

Le Mayday est un festival étudiant 100\% bénévole. Cet événement met l’accent sur l’expérimentation et l’innovation. En suivant cette démarche, l’édition 2024 a opéré de nombreux changements dans l’organisation du festival en montrant une ambition certaine sur le volet écologique et social.

\section{Nouveautés de l'édition 2024}

\paragraph{Pôle innovation écologique et sociale}
Plusieurs innovations ont été mises en place durant cette édition, notamment un pôle innovation écologique et sociale. Anciennement pôle village, cette complète restructuration a été la source de nombreux changements : 
\begin{itemize}
\item Charte générale d’engagements
\item Charte alimentaire, l'alimentation sur le festival était 100\% végétarienne,
\item Friperie solidaire
\item Table ronde
\item Goûter solidaire
\item Spectacle vivant
\end{itemize}

\paragraph{Repas maison pour les équipes}
Par ailleurs, et pour appliquer la charte alimentaire, une cuisine maison a été mise en place pour les bénévoles, prestataires et organisateurs avec le concours du CROUS. Pour des raisons techniques (notamment la gestion de multiples demandes et régimes spécifiques), nous n'avons pas encore pu mettre en place cette innovation pour les artistes. Cela fait partie des points que nous souhaitons initier pour les prochaines éditions.

\paragraph{Passage à 2 jours de festival}
Un passage à 2 jours de festival a aussi été réalisé. Il permet notamment d'offrir plus d'opportunités de sensibilisation, de toucher un public plus large et de faire se produire davantage d'artistes. La section suivante est dédiée à cette évolution.

\paragraph{Safe zone}
Comme nous avons pu en faire état précédemment dans ce bilan, une \og Safe zone \fg{} a été mise en place sur une temporalité identique à l'accueil du public sur site. Grand espace à l'écart de la foule, il avait pour vocation d'accueillir toutes les personnes qui le souhaiteraient. Une équipe de bénévoles, formés à la prévention des VSS\footnote{Violences sexistes et sexuelles} et informés des dispositifs que de potentielles victimes pouvaient solliciter, étaient en relation étroite avec la protection civile de Talence, dont le poste de secours était implanté à proximité. 

\section{Améliorations}
Cette édition s'est particulièrement distinguée, en dépassant plusieurs records établis sur les 5 premières années d'existence du festival. En effet, le nombre de partenaires du festival, le nombre de bénévoles impliqués les Jours J et dans l'organisation en amont. Toujours dans la dynamique d'offrir la possibilité au plus grand nombre d'artistes, notamment émergeant, de se produire sur les scènes du Mayday et permettre un accès gratuit aux arts, nous avons doublé le nombre de groupe accueillis.

Un changement notable a également pris place derrière dans les buvettes. En effet, depuis plusieurs années le Mayday travaillait avec une brasserie située à 40 km du festival. Avec cette évolution, notre nouveau partenaire se situe directement à Bordeaux (à 5 km des scènes). Ses engagements sont davantage en symbiose avec du festival.

\section{Passage à 2 jours de festival}

\paragraph{Objectifs du passage à 2 jours}
Mêlant plusieurs objectifs, le passage à 2 jours de festival a concouru à : 
\begin{itemize}
\item Développer le festival
\item Rationaliser le temps d'installation avec la durée d'exploitation
\item Ouvrir les scènes du Mayday à un nombre plus élevé d'artistes, permettant ainsi plus de diffusion de culture
\end{itemize}

\paragraph{Analyse de l'impact de cette évolution}
\subparagraph{Plan humain}
A posteriori, plusieurs éléments permettent de questionner de cette augmentation de voilure. En effet, sur le plan humain il apparaît que la charge de travail n'est pas proportionnellement supérieure, mais reste particulièrement conséquente. Elle se répartit comme tel : 
\begin{itemize}
\item Organisation : charge supplémentaire sur la programmation artistique, la gestion logistique (booking hôtel, coordination repas, backline) et administrative (contrats, déclarations)
\item Gestion jours J : davantage de déplacements et de facteurs à concilier
\end{itemize}

\subparagraph{Plan financier}
Dans la continuité, sur le plan financier :
\begin{itemize}
\item Nous constatons que l'évolution du coût entre la charge additionnelle relative à l'immobilisation du matériel et le temps de travail (doublé) des techniciens des éditions 2023 et 2024 est de l'ordre de $x1,5$\footnote{Nous observons que le chiffre ne semble pas proportionnel à l'augmentation du temps de travail, mais il s'explique par un coût d'immobilisation moins élevé} sur les périmètres son \& lumière, scène et backline. Il est important de noter que le niveau de rémunération des techniciens intervenant a été maintenu. Cela correspond à nos estimations.
\item La présence d'agents de sécurité privée a augmenté proportionnellement au nombre d'agents et au temps passé sur site. Cette donnée est en accord avec nos projections.
\item La programmation budgétaire de l'édition 2024 allouait 25 000 € pour payer les artistes de cette édition. Cette augmentation n'est pas proportionnelle à l'évolution du nombre de groupes\footnote{23 589 € dépensés en 2023}, elle correspond à 5 \%. Effectivement 25 749 € ont été consommés, soit 2\% de dépassement.
\item Sur le plan des recettes, les projections de bénéfices relatifs aux ventes de la buvette de concours avec l'augmentation de plusieurs subventions publiques avaient pour vocation de compenser le coût de cette soirée supplémentaire.
\end{itemize}
Le bilan financier de cette édition fait état d'un équilibre entre les recettes et les dépenses. L'augmentation à 2 jours de festival n'a pas eu d'impact délétère. Une partie ultérieure du rapport est dédiée à l'analyse du budget de cette édition.

\subparagraph{Installation}
Le bon déroulement de cette soirée additionnelle a nécessité une installation anticipée de l'ensemble du matériel. Ainsi, à la place de commencer usuellement à J-3 ou J-4, les premières installations et réceptions de matériels ont eu lieu à J-7. Par ailleurs, le festival ayant normalement lieu le samedi, le campus n'était pas occupé. Pour cette édition, nous avons du nous adapter pour ne pas gêner les activités universitaires habituelles, particulièrement le vendredi avec les balances. Cette adaptation a largement était accompagnée par l'université de Bordeaux\footnote{Notamment par le service scolarité du collège sciences et technologies (cours déplacés dans d'autres bâtiments), le bureau de la vie étudiante du campus de Talence et la direction des services à l'occupant Peixotto-Bordes.}.

\chapter{Retour d'expérience}

\section{Phase d'organisation (août 2023 - mai 2024)}
\vfill

\section{Phase d'installation (13 - 17 mai 2024)}

\paragraph{Livraisons}

Du vendredi 10 mai au vendredi 17 mai, le festival fut mis en place. Plusieurs installation lourdes étaient nécessaires. Elles regroupent notamment les livraisons pour : 
\begin{itemize}
\item les scènes,
\item les buvettes,
\item la technique (son et lumière),
\item les toilettes sèches, 
\item les barrières (HERAS et vauban),
\item et d'autre petit matériel (panneaux, électricité, ...).
\end{itemize}

\paragraph{Méthodologie}

L'ensemble des arrivées de matériel étaient enregistrées dans un tableur de suivi. Cela permettait que les véhicules soient systématiquement accueillis par un binôme, chargé d'ouvrir (ou de demander l'ouverture, le cas échéant) l'accès au véhicule, le guider à l'emplacement prévu et prévenir aux accidents (notamment avec les piétons et les cyclistes).

\paragraph{Opérations}

\subparagraph{Signalétique}

Afin de prévenir les usagers et voisins du festival, les équipes du festival ont procédé à une campagne d'affichage et de dépose de flyers dans les boîtes aux lettres des riverains. Par ailleurs, l'affichage des arrêté et de panneaux d'information sur les parking ayant été réalisés en amont, la fermeture des parkings fût aisée. Un affichage dédié aux personnes se déplaçant sur le campus (incluant notamment un itinéraire accessible aux personnes à mobilité réduite) avait été prévu, mais sa mise en place a été trop tardive. Nous notons pour les éditions suivantes, qu'un affichage la semaine précédente peut permettre de répondre aux problématiques d'information des usagers et de charge de travail des bénévoles durant l'installation. Grâce au concours du pôle patrimoine et environnement, un mail d'information avait été diffusé à la communauté universitaire (étudiants et personnels) du campus.

\subparagraph{Sécurité}

Plusieurs enjeux de sécurité coexistaient durant les temps d'installation et de mise en place du festival : 
\begin{itemize}
\item Sécurisation des livraisons,
\item Protection des usagers classiques du campus,
\item Stockage du matériel (nuit et intempéries).
\end{itemize} 

Le festival se déroulant sur le campus universitaire de sciences durant une période d'activité, l'organisation du festival devait s'adapter : 
\begin{itemize}
\item Niveau de bruit faible pendant les horaires d'enseignements ainsi que la nuit (voisinage),
\item Maintenir les accès et signaler les déviations, notamment PMR et vélo, entre les différents bâtiments et pour traverser le campus,
\item Partager les espaces intérieurs avec les utilisateurs habituels (services, enseignants, associations étudiantes, ...).
\end{itemize}

Pour cela, un étalement de l'installation a été mis en place. Habituellement commençant au milieu de la semaine, les livraisons ont commencées dès le vendredi précédent. Cela permettait particulièrement de mieux gérer les flux d'usagers du campus.

Par ailleurs, le soutien de l'université de Bordeaux au travers de son service scolarité du collège sciences et technologies était ici primordial. Il a permis le déplacement de nombreux enseignements à distance des espaces "à risques de nuisances" vers des lieux plus calme, à proximité.

Un arrêté du président de l'université de Bordeaux prévoyait, dès le début de la semaine, la fermeture de plusieurs axes de circulation (piéton, vélo et VL). Cela permettait de :
\begin{itemize}
\item Mettre en place un barriérage de sécurité autour du site (installation progressive suivant l'installation),
\item Ouvrir des voies d'accès sécurisée depuis et vers les bâtiments ainsi que les sorties du campus,
\item Faciliter la sécurisation des livraisons, dont les manœuvres pouvaient avoir lieu dans un environnement contrôlé,
\item Concourir à la réussite du travail des agents de sécurité en mission de gardiennage de nuit.
\end{itemize}

\subparagraph{Bénévoles}

Plusieurs périodes de présence de volontaires bénévoles avait étaient identifiées en amont (appel à bénévoles en janvier 2024). Soutien très important au bon déroulement des opérations d'installation, nous avons cependant noté plusieurs points : 
\begin{itemize}
\item Disponibilités des personnes bénévoles limitées durant la semaine.
\item Périodes pré-identifiées qui ne correspondent pas nécessairement au besoin réel, nécessité de flexibilité à J-7 de l'événement (en fonction de livraisons dont la temporalité évoluerait indépendamment de la volonté des organisateurs).
\end{itemize}

\section{Phase d'exploitation (17 - 18 mai 2024)}
\vfill

\section{Phase de démontage (19 mai - juin 2024)}

Particulièrement orienté sur la mise en place du festival et son bon déroulement, les organisateurs ont manqué de préparation pour la phase de démontage. Un planning théorique avait été préparé, mais n'a pas été mis à jour ni suivi.

Dans la même temporalité, les organisateurs n'ayant pas bénéficié d'un repos suffisant durant l'organisation et la tenue du festival\footnote{Une moyenne de 5h de repos par période du 24h, sur les 7 jours précédents a été observée.}, une faible productivité a été relevée.

\chapter{Analyse financière}

%\section{Bilan}
%
%\part{Opérationnel (17 mai - 18 mai 2024)}
%\section{Sécurité}
%
%
%\part{Désinstallation et rangement (19 mai - 7 juin 2024)}
%
%\part{Bilan moral}
%\chapter{Partenaires}
%
%\part{Bilan financier}
%\chapter{Analyse financière - passage à 2 jours}
%
%\subsection{subsection}
%\subsubsection{subsubsection}
%\paragraph{paragraph}
%\subparagraph{subparagraph}

\end{document}