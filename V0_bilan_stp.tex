\documentclass[12pt,a4paper]{report}
\linespread{1.2}

%Packages
\usepackage[utf8]{inputenc}
\usepackage[french]{babel}
\usepackage[hidelinks]{hyperref}
\usepackage{microtype}
\usepackage[T1]{fontenc}
\usepackage{helvet}
\renewcommand{\familydefault}{\sfdefault}
\usepackage{amsmath}
\usepackage{amsfonts}
\usepackage{amssymb}
\usepackage{graphicx}
\usepackage{lmodern}
\usepackage{array}
\usepackage{multirow}
\usepackage{threeparttable} %Pour footnote simili-utilisables dans les tableaux
\usepackage{soul} \usepackage{color} \newcommand{\hilight}[1]{\colorbox{yellow}{#1}}
\usepackage{geometry}
\geometry{hmargin=2cm,vmargin=1.5cm}

\AddThinSpaceBeforeFootnotes % à insérer si on utilise \usepackage[french]{babel}
\FrenchFootnotes % à insérer si on utilise \usepackage[french]{babel}

\renewcommand{\thefootnote}{\roman{footnote}}

% Titre
\title{Bilan moral du Festival Mayday \\ 6$^e$ édition - 2024}
\author{Équipe Mayday 2024}
\date{Vendredi 31 mai 2024}

\makeindex

\begin{document}

\maketitle

\begin{abstract} %Résumé
Le bilan moral de la 6ème édition du festival Mayday 2024 met en lumière les succès et les défis rencontrés lors de la réalisation de cet événement de deux jours. Malgré un contexte financier difficile hérité de l’édition précédente, les organisateurs ont réussi à équilibrer le budget grâce à une gestion rigoureuse et à l’anticipation des risques. L’expansion du festival sur deux jours a permis d’accueillir plus d’artistes et de proposer un programme plus riche.\\

L’engagement de plus de 200 bénévoles a été essentiel à la réussite du festival, ils ont réalisé des missions essentielles à la tenue de l'événement. Des améliorations ont été apportées, notamment en matière de restauration avec une offre 100\% végétarienne et une cuisine maison pour les équipes. Le festival a également mis en place une \og Safe zone \fg{} pour garantir la sécurité et le bien-être des festivaliers.\\

Malgré une fréquentation légèrement inférieure aux estimations en raison de la météo, le bilan financier de cette édition est équilibré. L’augmentation du nombre de partenaires et la tenue de buvettes lors d'événements annexes ont permis de compenser les coûts supplémentaires liés à l’extension du festival sur deux jours.\\

Cependant, des défis subsistent, notamment en ce qui concerne la répartition des tâches au sein de l’équipe organisatrice et la gestion de la phase de démontage. Le bilan souligne également l’importance d’une communication renforcée pour mieux faire connaître les dispositifs de prévention et les engagements "transitions" mis en place lors du festival. Dans l’ensemble, le bilan moral de la 6$^e$ édition du festival Mayday est positif, mettant en évidence la capacité de la nouvelle équipe à relever les défis et à innover pour offrir un événement culturel de qualité, accessible à tous et respectueux de l’environnement.
\end{abstract}

\tableofcontents

\newpage

\section*{Avant-propos}
\addcontentsline{toc}{section}{Avant-propos}

Le Mayday Festival est une manifestation culturelle. Mêlant musique et sensibilisation aux transitions socio-environnementales, ce festival s'articule autour d'objectifs durables et ambitieux :
\begin{itemize}
\item Ouverture de l’accès à la culture à toutes et tous,
\item Sensibilisation du plus grand nombre aux transitions environnementales et sociétales,
\item Être un laboratoire d’incubation, chef de fil en matière d’événementiel éco-responsable,
\item Mettre en avant des artistes locaux, émergents et étudiants,
\item Faire du campus non plus seulement un lieu de passage et d'étude, mais aussi un lieu de vie que les étudiants s'approprient,
\item Favoriser l’ouverture du monde académique aux arts vivants et vers le territoire métropolitain.\\
\end{itemize}

Dans le but d’être ouvert à plus de public et de permettre le passage de davantage d’artistes, la sixième édition du festival se déroulait sur 2 jours. Cette évolution ambitieuse, mais cohérente avec les objectifs, fera l’objet d’une analyse dans ce rapport. Véritable vitrine et vecteur de l'engagement étudiant sur les campus bordelais, le Mayday est une expérience rassemblant de nombreux partenaires, acteurs du territoire, services publics et étudiants. Plusieurs périodes se sont succédé tout au long de la vie du projet ; ces séquences seront détaillées dans le rapport, avant les bilans et les conclusions tirées de cette édition.\\

Ce bilan moral est l’occasion pour les organisateurs de revenir sur l’édition 2024 du Mayday. Commençant par le déroulement de l'événement en lui-même, nous poursuivrons par une partie dédiée à l'équipe organisatrice, puis par un retour sur les innovations mises en œuvre sur cette édition. Nous proposerons une analyse des points forts et des fragilités permettant de mieux appréhender les évolutions à mettre en place sur les prochaines éditions. Un retour d'expérience par séquence de l'événement (organisation, installation, exploitation, démontage) sera ensuite exposé. Ce bilan sera clôturé par une discussion sur les différents horizons pour le festival Mayday.

\subsection*{Projet unique}
\addcontentsline{toc}{subsection}{Projet unique}

Le Mayday est un projet unique : il s'agit du plus grand festival gratuit en France et porté par des étudiants, tous issus de différentes formations. Ces bénévoles se réunissent et s'investissent au sein de ce projet pendant une année (de août 2023 pour la préparation de l'édition 2024, à juillet 2024 pour la rédaction du bilan de l’événement). Cet engagement apparaît très fort au regard des autres activités dans lesquelles ces personnes sont impliquées : étude, emploi étudiant, autres mandats associatifs ou représentatifs.\\

\sloppy Le temps investi est cependant sanctionné par l'acquisition de compétences, connaissances et savoir-être importants. À titre d'exemple, ces étudiants développent des compétences de gestion de projet (rétroplanning, mailing, suivi de tâches, …), organisationnelles (nombreuses réunions, suivi de relations avec des partenaires et prestataires) et relationnelles (communication par mail, téléphonique, réunions). C’est aussi une expérience concrète de travail en équipe, où la communication et l'entraide sont primordiales. Enfin, c’est une introduction, parfois même une découverte, au monde professionnel durant laquelle les organisateurs vont travailler avec les établissements partenaires, un grand nombre d'entreprises, d’associations et de collectivités locales.\\

L’engagement bénévole des étudiants et les soft-skills\footnote{Ce sont des aptitudes personnelles et relationnelles qui permettent de mieux interagir avec les autres et de s'adapter à différentes situations. Ils se distinguent des hard skills, qui sont les compétences techniques et les connaissances spécifiques à un métier.} développés lors de l’expérience Mayday sont reconnus et valorisés par les employeurs.\\

Ce festival est un évènement unique. De par son lien avec les universités bordelaises, le CROUS, les collectivités territoriales. Il fait rayonner l'ensemble de ses partenaires au travers de ses valeurs. Il s’agit d’un évènement fédérateur pour toute la communauté universitaire et métropolitaine.

\subsection*{Contexte financier}
\addcontentsline{toc}{subsection}{Contexte financier}

\paragraph*{État de la trésorerie suite au Mayday 2023 (mai - juillet 2023)}
La sixième édition du festival a été réalisée dans un contexte financier difficile pour l’association. En effet, le bilan de l'édition 2023 a fait état d'une gestion financière défaillante, marquée notamment par un déficit de 36 733,11 €. Cette situation met en évidence un manque de maîtrise des dépenses, de contrôle par les organes compétents\footnote{Définit dans les statuts de l'association M-Tech, il s'agit du conseil d'administration et du bureau} et une programmation budgétaire insuffisamment étayée, ne permettant pas d'anticiper certains écueils financiers (plusieurs subventions sollicitées non acquises). Caractérisant notamment le manque de maîtrise des dépenses, le maintien de certains engagements contractuels, malgré leur poids financier et les contraintes techniques croissantes, a contribué à la situation déficitaire.

\paragraph*{Décision prises dans le périmètre trésorerie (septembre - octobre 2023)}
L'association a vu ses liquidités significativement consommées, le passif exigible ne semblant pas excéder l'actif qui serait disponible mensuellement, d'après la programmation établie par les personnes responsables des comptes, une échéance mensuelle des montants dûs a été négociée pour une période de 10 mois.

\paragraph*{Engagements et enjeux pour l'édition du Mayday 2024 (juillet - septembre 2023)}
Il convient donc d'établir que la situation de l'association au moment du lancement du projet \og Mayday 2024 \fg{}  devait être stabilisée par une gestion maîtrisée, contrôlée par les organes compétents et anticipant de potentiels aléas (risques météorologiques, de sécurité ou sanitaires).

\newpage
\section*{Remerciements}
\addcontentsline{toc}{section}{Remerciements}

L’équipe du Mayday tient à remercier chaleureusement toutes les personnes qui ont contribué à la réussite de cette sixième édition. Nous sommes particulièrement reconnaissants envers :
\begin{itemize}
\item Les bénévoles, dont l’engagement et le dévouement sont essentiels au bon déroulement de cet événement. 
\item Les partenaires, dont le soutien financier et matériel a permis de concrétiser les ambitions des organisateurs et d’offrir un festival de qualité, gratuit et accessible à tous.
\item Les services de l’université de Bordeaux, dont l’accompagnement et la collaboration ont été précieux pour la réalisation du festival sur le campus.
\item Le CROUS Bordeaux-Aquitaine, pour son accompagnement, pour le partenariat annuel noué avec l’association, pour son aide précieuse dans la mise en place de la restauration et pour sa participation active à l’événement.
\item Les collectivités territoriales, qui ont soutenu cette initiative et ont permis de renforcer le lien entre l’université et la communauté locale.
\item Les artistes, qui ont partagé leur talent et leur passion avec le public.
\item Les associations, qui ont enrichi le festival par leurs actions et leurs messages de sensibilisation.
\item Les food trucks, qui ont proposé une offre alimentaire végétarienne, variée et de qualité, en accord avec nos valeurs écologiques.
\item Le public, qui a répondu présent malgré la météo capricieuse et qui a contribué à l’ambiance festive et conviviale du festival.\\
\end{itemize}

Elle tient également à remercier tous ceux qui, de près ou de loin, ont apporté leur pierre à l’édifice et ont fait de cette sixième édition un moment inoubliable. L’équipe du Mayday espère vous retrouver encore plus nombreux l’année prochaine pour célébrer la musique et l’engagement étudiant !

\chapter{Déroulement de l'événement}
\section{Vendredi 17 mai}

Le festival a ouvert ses portes à 18h50, avec un retard de 20 minutes dû à un imprévu de dernière minute (déclenchement inopiné de l'alarme incendie dans un bâtiment). Malgré ce contretemps, l'événement s'est déroulé avec succès.\\

Le DJ set de Ekograal a commencé comme prévu à 18h30, mais le public ne pouvant pas entrer sur site, il s'est brièvement arrêté. Le set a ensuite recommencé lorsque le public a pu pénétrer sur le site (18h50), ce qui a créé un retard dans l'enchaînement de la soirée. Un temps entre chaque concert avait été prévu en cas de besoin, mais sa durée n'était pas suffisante pour permettre de rattraper le retard. Un décalage des sets a permis à ce que chaque artiste puisse tout de même jouer. La fin des concerts à eu lieu à 00h10, soit 10 minutes après l’heure initialement prévue.\\

Les deux bars principaux ont fonctionné de manière ininterrompue de 18h50 à 0h et ont servi des bières blondes et blanches de la brasserie LaLune et de la limonade bio de la Maison Meneau. De l’eau était servie gratuitement au niveau d’un troisième bar et dans la Safe-Zone.\\

Sur le volet restauration, 7 food trucks proposant une offre 100\% végétarienne étaient présents. De plus, un stand de pâtisserie végan était tenu par Le Maraudeur : un étudiant qui s'engage au travers de pâtisseries vegans, peu onéreuses et sans allergènes.\\

Plusieurs partenaires du festival étaient venus partager leurs actions lors de cette soirée : 
\begin{itemize}
\item la société Karos, qui gère une plateforme de covoiturage,
\item le CROUS Bordeaux-Aquitaine, qui est venu communiquer sur la finale nationale du tremplin musical pulsations qui avait lieu la semaine suivante.\\
\end{itemize}

Au niveau de l’entrée, des inspections visuelles, palpations et détections de métaux ont permis de prévenir les risques sur le festival. Il n'y a eu aucune évacuation et la protection civile n’a fait qu’une seule prise en charge (crise d'épilepsie, patiente avec historique). Le placement de la safe zone, à l'écart de la foule, a permis qu'elle soit un lieu confidentiel. Seul un petit nombre de festivaliers ont fréquenté la zone de sensibilisation. Des binômes mobiles de bénévoles formés faisaient des rondes sur le reste du site afin de prévenir les problèmes.\\

175 bénévoles ont apporté leur concours lors de cette soirée. À l'issue d'une formation dispensée plus tôt dans le mois, ils ont occupé des postes variés. Leur arrivée sur site s'est faite au compte goutte, sur toute la journée. Avant l'ouverture, un briefing général a eu lieu. Il a permis de refaire le tour du site et des missions sur lesquelles les bénévoles seront amenés à contribuer. Les retours sur leur expérience ont été particulièrement positifs, à l'exception de certains postes où l'activité menée pouvait être passive. En guise de remerciement, nous offrions 2 consommations/bénévole/jour, les repas ainsi que le T-shirt bénévole Mayday.\\

Une gestion des déchets ambitieuse fût mise en œuvre cette année. Accompagné par des brigades de bénévoles, plusieurs types de déchets avaient été identifiés. Ils ont pu être quantifiés et ces informations sont partagées dans le rapport d'activité de la sixième édition. De plus, une enquête a été menée auprès d'un échantillon de festivaliers. Elle nous a permis de récolter des données très intéressantes pour le pilotage du festival (mode de venue, canal de communication, ...). Ces données traitées sont également disponibles dans le rapport d'activité.\\

L'évacuation du public s'est réalisée sans difficulté, et le site était vide à 0h25. Il n'y a pas eu de formation d'attroupement sur la voie publique grâce au concours des agents TBM et de bénévoles qui ont immédiatement conduit les festivaliers vers les transports en commun. L'interdiction de circulation sur l'esplanade de la bibliothèque sciences et techniques et des voies y menant a permis une gestion optimale des flux.\\

Cette soirée fût cependant teintée par plusieurs petites et courtes averses pendant les concerts. Environ 1000 litres de bières ont été vendues.

\section{Samedi 18 mai}

Les portes du festival ont ouvert à 14h15 (15 minutes après l'horaire annoncé). En conséquence, la table ronde prévue a été repoussée. L'organisation ayant envisagé ce potentiel retard, la suite du programme n'a pas été affectée.\\

Jusqu'à 18h, se sont tenus plusieurs temps forts : table ronde, labioratoire\footnote{Stands de sensibilisation}, friperie, jeux divers, spectacle vivant (en partenariat avec l'IUT Bordeaux Montaigne) et goûter solidaire. Nous avons eu le plaisir de recevoir la visite du Maire de Talence et du Vice-président en charge de la vie étudiante et de la vie de campus de l'université de Bordeaux.\\

Une grosse averse est survenue aux alentours de 18h20, et a duré jusqu'à 18h40. La réactivité des équipes et l'anticipation ont permis de limiter les avaries techniques. Sur le même modèle que la veille, les concerts suivants ont été légèrement repoussés mais le retard a été absorbé par le temps de battement réduit entre les sets.\\

Présent en nombre plus élevé lors de cette soirée, le public est évacué à la fin des concerts et se disperse sans heurt. A nouveau, le concours des agents TBM a facilité la dispersion. Le site est vide à 1h30.\\

Durant l'après-midi, les buvettes ont fonctionné à 50\% de leur capacité. En effet, la demande étant moins élevée nous avons préféré que les bénévoles profitent de temps libre.\\


Les mêmes food-trucks que la veille étaient présents, les ventes furent légèrement supérieures.\\

Sur la soirée, le CROUS Bordeaux-Aquitaine fût à nouveau présent.\\

Près de 200 bénévoles étaient volontaires pour ce deuxième soir de concerts.\\

Sur le périmètre sécurité/secours/prévention, un dispositif plus conséquent que la veille a été déployé. Aucune victime ne fut à déplorer, et la safe zone connut une fréquentation légèrement supérieure.\\

Dans la continuité du dispositif de gestion des déchets mis en place le vendredi, un dispositif amélioré suivant les habitudes et besoins observés a été mis en place.

\section{Affluence}
La fréquentation du festival a été suivie par des bénévoles postés après l'entrée du festival. Deux données ont ainsi été récoltées : les pics de fréquentation (un par séquence) et le nombre d'entrées (sans distinction si un visiteur entre à de multiples reprises). Le tableau suivant présente les pics d'affluence pour chaque séquence :
\begin{table}[h!]
\centering
\begin{tabular}{|c|c|c|}
\hline
Vendredi soir & Samedi après-midi & Samedi soir \\
\hline
4000 & 400 & 8000\\
\hline
\end{tabular}
\caption{Tableau récapitulatif des pics de fréquentation de l'édition 2024 du festival Mayday}
\end{table}

Sur l'ensemble des deux jours, 17 000 entrées ont été comptabilisées. Un bilan en légère baisse par rapport à l'édition précédente, mais justifié malgré une météo incertaine, notamment le vendredi soir. L'affluence est restée satisfaisante, témoignant de l'attrait du public pour la programmation éclectique et les animations proposées. Le samedi après-midi, dédié aux activités de sensibilisation et aux spectacles, a attiré un public familial, tandis que la soirée a vu une affluence plus importante, notamment grâce à la réputation grandissante du festival.

\chapter{Équipe organisatrice}

Une équipe motivée et dynamique est l'essence du festival.\\

L’équipe organisatrice est composée de 23 étudiants, venant de formations différentes. Ils ont fait le choix de s’investir bénévolement pour ce projet ambitieux et pour des valeurs communes.\\

Il s'agit de la première année où l'équipe chargée de l'organisation est aussi nombreuse. Une organisation répartissant les missions par périmètre thématique a été mise en œuvre. A posteriori, il apparaît que faire évoluer l'organisation pour obtenir une répartition des tâches plus claire, faciliter le suivi et le management de profils néophytes est indiqué.\\

Plusieurs points forts sont à souligner : 
\begin{itemize}
\item engagement fort de l'équipe,
\item motivation et dynamique.\\
\end{itemize}

Cependant des points faibles le sont aussi : 
\begin{itemize}
\item ambiance en réunion durant une partie de l'année,
\item difficulté à mettre en place au cours de l’année une attitude adaptée en fonction des temps\footnote{une attitude de travail en réunion et une attitude pouvant être plus amicale en dehors},
\item une équipe majoritairement néophyte en gestion de projet, en organisation d’événement et en associatif.\\
\end{itemize}

Sans qu'ils entrent nécessairement dans l'une ou l'autre de ces catégories, les points suivant ont aussi été identifiés :
\begin{itemize}
\item \textbf{valeurs et ambitions du projet} : projet inspirant, stimulant, partage et transmission,
\item \textbf{typologie} : événement festif, culturel, grande ampleur,
\item \textbf{engagement} : 1 an, dépassement de soi, professionnalisation (découverte de vocation), prise de responsabilité, compétences à acquérir, expérience unique, pas ou peu de passation avec l'équipe précédente,
\item \textbf{impact de l'événement} : sentiment d'utilité, sensibilisation, éducation populaire, médiation,
\item \textbf{équipe} : esprit d'équipe, sentiment d'appartenance, équipe motivée et volontaire.\\
\end{itemize}

\newpage
Cette année, l'équipe organisatrice du Mayday était composée par :

\begin{table}[h!]
\centering
\begin{tabular}{|m{3cm}|m{10cm}|}
\hline
Pôle & Prénom \\ 
 \hline\hline
Transverse & \textbf{Louis Delignac}\\ 
\hline
\multirow{2}{10em}{Sécurité technique prévention} & \textbf{Sacha Duperret} \\ 
& Antoine Philippeau, Ebène Bilé, Maël Avadian, Flavio Arella\\ 
 \hline
 \multirow{2}{5em}{Innovation écologique et sociale} & \textbf{Baptiste Royau} \\ 
& Katarina Montenier, Manon Bénard, Eléa Peynaud, Leelou Cateigbou\\
 \hline
 \multirow{2}{5em}{Trésorerie} & \textbf{Aurélien Gauthier} \\ 
& Victor Lamare\\
 \hline
 \multirow{2}{5em}{Artistes} & \textbf{Leïla Karim} \\
& Lucile Guerin\\
 \hline
\multirow{2}{5em}{Bénévoles} & \textbf{Nabil Es-Selymy} \\ 
& Anessa Diallo\\
 \hline
\multirow{2}{7em}{Communication} & \textbf{Alice Conti} \\ 
& Salma Rabion, Nassim Sellal, Julie Fourneyron\\
 \hline
 \multirow{1}{5em}{Logistique} & \textbf{Arthur Vienot} \\ 
 \hline
 \multirow{1}{5em}{Restauration} & \textbf{Lucas Brouet} \\ 
 \hline
\end{tabular}
\caption{Tableau récapitulatifs des membres organisateurs du Mayday 2024}
\label{tab:team}
\end{table}

\textit{Les responsables de pôle sont identifiés en gras.}

\chapter{Innovations de l'édition 2024}

Le Mayday est un festival étudiant 100\% bénévole. Cet événement met l’accent sur l’expérimentation et l’innovation. En suivant cette démarche, l’édition 2024 a opéré de nombreux changements dans l’organisation du festival en montrant une ambition certaine sur le volet écologique et social.

\section{Nouveautés de l'édition 2024}

\paragraph{Pôle innovation écologique et sociale}
Pour ancrer son engagement dans les transitions, le Mayday s’est doté d’un pôle Innovation écologique et sociale (IES). Anciennement pôle village, ces missions sont diverses : veille du respect des valeurs du festival, mise en place des différents temps forts du samedi après-midi, gestion des déchets, calcul du bilan carbone.\\

La création de ce nouveau pôle a été la source de nombreuses nouveautés comme la mise en place d’une Charte générale d’engagements et d’une Charte alimentaire imposant l'alimentation 100\% végétarienne sur le festival.\\

Les temps forts du samedi après-midi ont aussi étés sujets à des innovations, pour la première fois de l’histoire du Mayday, en plus du traditionnel village associatif, une friperie solidaire après une collecte de vêtements de 5 mois, une table ronde avec trois intervenants et une médiatrice, un goûter finançant trois associations et une spectacle vivant animé par des étudiants de la formation Animation Socioculturelle et culturelle de l’IUT Bordeaux Montaigne ont eu lieu.\\

Une attention particulière a été portée sur la gestion des déchets et sur le bilan carbone de l’événement. Nous aurons l’occasion de revenir dessus plus tard dans ce document.

\paragraph{Repas maison pour les équipes et patente pour les food-trucks}
Par ailleurs, une cuisine maison a été mise en place pour les bénévoles, prestataires et organisateurs avec le concours du CROUS. Habituellement, ce sont les food-trucks qui restauraient les bénévoles en échange de leur présence sur le festival et les prestataires étaient nourris aux frais de l’association. Cette année les food-trucks payaient un prix fixe pour la location d'un emplacement. Ce changement a permis d’apporter des ressources supplémentaires au festival. Pour des raisons techniques (notamment la gestion de multiples demandes et régimes spécifiques), cette innovation n'a pas pu être mise en place pour les artistes.

\paragraph{Passage à 2 jours de festival}
Permettant d'offrir plus d'opportunités de sensibilisation, de toucher un public plus large et de faire se produire davantage d'artistes, la durée du festival a été portée à 2 jours. Nous aurons l’occasion de revenir sur cette évolution dans une partie qui lui est dédiée plus bas.

\paragraph{Safe zone}
Une “Safe zone” a été mise en place pendant toute la durée d’ouverture du festival. Grand espace à l'écart de la foule, il avait pour vocation d'accueillir toutes les personnes qui le souhaiteraient. Une équipe de bénévoles, formés à la prévention des VSS\footnote{violences sexistes et sexuelles} et informés des dispositifs que de potentielles victimes pouvaient solliciter, étaient en relation étroite avec la protection civile de Talence, via le poste de secours était implanté à proximité. De plus, des binômes mobiles permettaient une vigilance portée sur l’ensemble du site.

\section{Améliorations}
Cette édition s'est particulièrement distinguée en dépassant plusieurs records, établis sur les 5 premières éditions du festival. En effet, le nombre de partenaires du festival, le nombre de bénévoles impliqués les Jours J et dans l'organisation en amont ont étés nettement supérieurs. Toujours dans la dynamique d'offrir la possibilité au plus grand nombre d'artistes, notamment émergeant, de se produire sur les scènes du Mayday et permettre un accès gratuit aux arts, le nombre de groupes accueillis a été doublé.\\

En 2024, 84 structures partenaires ont été recensées :
\begin{itemize}
\item 28 associations,
\item 34 entreprises,
\item 3 institutions\footnote{université de Bordeaux, université Bordeaux Montaigne, CROUS Bordeaux-Aquitaine},
\item 11 services universitaires\footnote{Somme pour les 2 universités bordelaises},
\item 5 collectivités,
\item 1 service départemental,
\item 2 services déconcentrés.\\
\end{itemize}

Un changement notable a également pris place derrière dans les buvettes. En effet, depuis plusieurs années le Mayday travaillait avec une brasserie située à 40 km du festival. Avec cette évolution, notre nouveau partenaire se situe directement à Bordeaux (à 5 km des scènes). Ses engagements sont davantage en symbiose avec le festival.

\section{Passage à 2 jours de festival}

\paragraph{Objectifs du passage à 2 jours}
Mêlant plusieurs objectifs, le passage à 2 jours de festival a concouru à : 
\begin{itemize}
\item Développer le festival,
\item Rationaliser le temps d'installation avec la durée d'exploitation,
\item Ouvrir les scènes du Mayday à un nombre plus élevé d'artistes, permettant ainsi plus de diffusion de culture.
\end{itemize}

\paragraph{Analyse de l'impact de cette évolution}
\subparagraph{Plan humain}
A posteriori, plusieurs éléments permettent de questionner cette augmentation de voilure. En effet, sur le plan humain il apparaît que la charge de travail n'est pas proportionnellement supérieure, mais augmente de manière conséquente. Elle se répartit comme tel : 
\begin{itemize}
\item \textbf{Organisation} : charge supplémentaire sur la programmation artistique, la gestion logistique (booking hôtel, coordination repas, backline) et administrative (contrats, déclarations),
\item \textbf{Gestion jours J} : davantage de déplacements et de facteurs à concilier.
\end{itemize}

\subparagraph{Plan financier}
Dans la continuité, sur le plan financier :
\begin{itemize}
\item Nous constatons que l'évolution du coût entre la charge additionnelle relative à l'immobilisation du matériel et le temps de travail (doublé) des techniciens des éditions 2023 et 2024 est de l'ordre de x$1,5$\footnote{Nous observons que le chiffre ne semble pas proportionnel à l'augmentation du temps de travail, mais il s'explique par un coût d'immobilisation moins élevé} sur les périmètres son \& lumière, scène et backline. Il est important de noter que le niveau de rémunération des techniciens intervenant a été maintenu. Cela correspond à nos estimations.
\item La présence d'agents de sécurité privée a augmenté proportionnellement au nombre d'agents et au temps passé sur site. Cette donnée est en accord avec nos projections.
\item La programmation budgétaire de l'édition 2024 allouait 25 000 € pour payer les artistes de cette édition. Cette augmentation n'est pas proportionnelle à l'évolution du nombre de groupes\footnote{23 589 € dépensés en 2023}, elle correspond à 5 \%. Effectivement 25 749 € ont été consommés, soit 2\% de dépassement.
\item Sur le plan des recettes, les projections de bénéfices relatifs aux ventes de la buvette de concours avec l'augmentation de plusieurs subventions publiques avaient pour vocation de compenser le coût de cette soirée supplémentaire.\\
\end{itemize}
Le bilan financier de cette édition fait état d'un \textbf{équilibre entre les recettes et les dépenses}. \textbf{L'augmentation à 2 jours de festival n'a pas eu d'impact délétère}. Une partie ultérieure du rapport est dédiée à l'analyse du budget de cette édition.

\subparagraph{Installation}
Le bon déroulement de cette soirée additionnelle a nécessité une installation anticipée de l'ensemble du matériel. Ainsi, à la place de commencer usuellement à J-3 ou J-4, les premières installations et réceptions de matériels ont eu lieu à J-7. Par ailleurs, le festival ayant normalement lieu le samedi, le campus n'était pas occupé. Pour cette édition, nous avons dû nous adapter pour ne pas gêner les activités universitaires habituelles, particulièrement le vendredi avec les balances. Cette adaptation a largement était accompagnée par l'université de Bordeaux\footnote{notamment par le service scolarité du collège sciences et technologies (cours déplacés dans d'autres bâtiments), le bureau de la vie étudiante du campus de Talence et la direction des services à l'occupant Peixotto-Bordes}.

\chapter{Retour d'expérience}

\section{Phase d'organisation (août 2023 - mai 2024)}
\subsection{Recrutement}
De nombreux organisateurs d'éditions précédentes du festival ayant fini leurs études, l'équipe  s'est renouvelée en grande partie en fin d'année universitaire 2023.
Le recrutement s'est organisé via plusieurs canaux : 
\begin{itemize}
\item Réseau proche,
\item Annonce en ligne.
\end{itemize}

\subsubsection{Réseau proche}
Au travers d'expériences passées, les dirigeants nouvellement en poste avaient pu interagir avec d'autres étudiants. De nombreux profils ont étés identifiés, cependant l'ampleur du projet, l'investissement nécessaires et le contexte relatif à l'association ont étés des facteurs limitant.

\subsection{Pôles}
\subsubsection{Enjeux}
La répartition des tâches d'organisation du festival s'est construite selon plusieurs enjeux :
\begin{itemize}
\item Coordination du projet,
\item Sécurisation de l'événement,
\item Déclarations administratives,
\item Programmation artistique,
\item Matériel,
\item Électricité,
\item Sensibilisation,
\item Prévention,
\item Bilan carbone,
\item Communication,
\item Partenariat,
\item Gestion des bénévoles,
\item Gestion financière,
\item Restauration des équipes,
\item Restauration des festivaliers. \\
\end{itemize}

\subsubsection{Version finale}
Pour répondre aux besoins énoncés précédemment, une organisation en pôles thématiques a été mise en place. En voici une liste (ordre alphabétique) : 
\begin{itemize}
\item Administratif
\item Artistes
\item Bar
\item Bénévoles
\item Communication
\item Innovation écologique et sociale
\item Logistique
\item Partenariats
\item Presse
\item Restauration
\item Sécurité technique prévention
\item Transverse
\item Trésorerie
\end{itemize}

Le tableau \ref{tab:team} en présente les membres.

\subsection{Interactions}
Pour organiser cette $6^e$ édition, de nombreux échanges entre les différents pôles (passant notamment par leur responsable, chargé de la coordination des missions du pôle) ont été réalisés. Ils sont catégorisés en plusieurs types, détaillés dans les sections suivantes.

\subsubsection{Administratif}
Dans le cadre de déclarations, la mise en commun d'informations (régulièrement actualisées) était nécessaire. Elle comprenaient plusieurs volets :
\begin{itemize}
\item Implantation (plans, visites de site),
\item Besoins spécifiques (électricité, barrières, accès),
\item Risques spécifiques,
\item Utilisations particulières de matériel (non prévu dans l'usage préconisé),
\item Installation de CTS (chapiteaux, tentes, structures),
\item Plans alternatifs (météorologie),
\item Suivi des dates limites de dépôts, échanges avec les autorités, suivi des nouvelles réglementations.
\end{itemize}

\subsubsection{Communication et partenariats}
Dans le but de mettre en avant le travail fait bénévolement par ses membres, plusieurs missions de communication avaient étés identifiées :
\begin{itemize}
\item l'événement,
\item les engagements,
\item la gestion de projet,
\item les manifestations annexes (participations à des concours, tenue de buvettes, ...).\\
\end{itemize}

Pour cela, des échanges réguliers et la présence de l'équipe de communication en réunion hebdomadaire permettaient une remontée conforme des avancées et de la direction prise par le projet auprès du public.\\

Sur le plan partenarial (public professionnel, politique, administration), la mise en avant des enjeux de la manifestation, ainsi que les engagements humains derrière la façade festive se faisait notamment au travers d'une couverture photo/vidéo de nos interventions (séminaires, demandes de subventions, réunions) et de publications sur Linkedin.


\subsubsection{Montage de projet}
De manière plus générale, l'ensemble du montage du projet nécessitait des échanges globaux. Des point hebdomadaires étaient organisés, permettant que l'ensemble de l'équipe ait le même niveau d'information et de voter démocratiquement les orientations prises.

\section{Phase d'installation (13 - 17 mai 2024)}

\paragraph{Livraisons}

Du vendredi 10 mai au vendredi 17 mai, le festival fut mis en place. Plusieurs installation lourdes étaient nécessaires. Elles regroupent notamment les livraisons pour : 
\begin{itemize}
\item les scènes,
\item les buvettes,
\item la technique (son et lumière),
\item les toilettes sèches, 
\item les barrières et la signalétique,
\item et d'autre matériel.
\end{itemize}

\paragraph{Méthodologie}

L'ensemble des arrivées de matériel étaient enregistrées dans un tableau de suivi. Cela permettait que les véhicules soient systématiquement accueillis par un binôme, chargé d'ouvrir (ou de demander l'ouverture, le cas échéant) la voie au véhicule, le guider à l'emplacement prévu et prévenir aux accidents (notamment avec les piétons et les cyclistes).

\paragraph{Opérations}

\subparagraph{Signalétique}

Afin de prévenir les usagers et voisins du festival, les équipes du festival ont procédé à une campagne d'affichage et de dépose de flyers dans les boîtes aux lettres des riverains. Par ailleurs, l'affichage des arrêté et de panneaux d'information sur les parking ayant été réalisés en amont, la fermeture des parkings fût aisée. Un affichage dédié aux personnes se déplaçant sur le campus (incluant notamment un itinéraire accessible aux personnes à mobilité réduite) avait été prévu, mais sa mise en place a été trop tardive. Il apparaît, pour les éditions suivantes, qu'un affichage la semaine précédente peut permettre de répondre aux problématiques d'information des usagers et de charge de travail des bénévoles durant l'installation. Grâce au concours du pôle patrimoine et environnement, un mail d'information avait été diffusé à la communauté universitaire (étudiants et personnels) du campus.

\subparagraph{Sécurité}

Plusieurs enjeux de sécurité coexistaient durant les temps d'installation et de mise en place du festival : 
\begin{itemize}
\item Sécurisation des livraisons,
\item Protection des usagers du campus,
\item Stockage du matériel (nuit et intempéries).\\
\end{itemize}

Le festival se déroulant sur le campus universitaire de sciences durant une période d'activité, l'organisation du festival devait s'adapter : 
\begin{itemize}
\item Niveau de bruit faible pendant les horaires d'enseignements ainsi que la nuit (voisinage),
\item Maintenir les accès et signaler les déviations, notamment PMR et vélo, entre les différents bâtiments et pour traverser le campus,
\item Partager les espaces intérieurs avec les utilisateurs habituels (services, enseignants, associations étudiantes, ...).\\
\end{itemize}

Pour cela, un étalement de l'installation a été mis en place. Habituellement commençant au milieu de la semaine, les livraisons ont commencées dès le vendredi précédent. Cela permettait particulièrement de mieux gérer les flux d'usagers du campus.\\

Par ailleurs, le soutien de l'université de Bordeaux au travers de son service scolarité du collège sciences et technologies était ici primordial. Il a permis le déplacement de nombreux enseignements à distance des espaces "à risque de nuisance" vers des lieux plus calme, à proximité.\\

Un arrêté du président de l'université de Bordeaux prévoyait, dès le début de la semaine d'installation, la fermeture de plusieurs axes de circulation (piéton, vélo et VL). Cela permettait de :
\begin{itemize}
\item Mettre en place un barriérage de sécurité autour du site (installation progressive suivant l'installation),
\item Ouvrir des voies d'accès sécurisée depuis et vers les bâtiments ainsi que les sorties du campus,
\item Faciliter la sécurisation des livraisons, dont les manœuvres pouvaient avoir lieu dans un environnement contrôlé,
\item Concourir à la réussite du travail des agents de sécurité en mission de gardiennage de nuit.\\
\end{itemize}

\subparagraph{Bénévoles}

Plusieurs périodes de présence de volontaires bénévoles avait étaient identifiées en amont (appel à bénévoles en janvier 2024). Soutien très important au bon déroulement des opérations d'installation, plusieurs points ressortent cependant : 
\begin{itemize}
\item Disponibilités des personnes bénévoles limitées durant la semaine,
\item Périodes pré-identifiées qui ne correspondent pas nécessairement au besoin réel, nécessité de flexibilité à J-7 de l'événement (en fonction de livraisons dont la temporalité évoluerait indépendamment de la volonté des organisateurs).
\end{itemize}

\section{Phase d'exploitation (17 - 18 mai 2024)}

L'utilisation de talkies-walkies a grandement facilité la communication entre les différentes équipes, permettant une coordination efficace en temps réel. Il est cependant à relever que ce mode de communication n’a pas été utilisé de manière optimale. Une formation approfondie et une locution correcte pourraient permettre d’en faire un meilleur usage.\\


Le débit sortant étant trop faible pour suivre le rythme des arrivées, notamment aux moments de pics, il apparaît qu’un renforcement est nécessaire sur cette partie.\\

La mise à disposition de salles au rez-de-chaussée pour la logistique a permis de fluidifier les opérations en dehors des horaires d'ouverture du festival. Une installation optimale pourrait se faire avec un espace dédié, sécurisé et protégé en dehors des bâtiments.\\

L'accessibilité pour les personnes à mobilité réduite (PMR) a été globalement satisfaisante, mais des améliorations sont encore possibles.\\

Un incident de gazage à l'entrée a été rapidement géré, bien qu'il soulève des questions concernant la collaboration avec la société d'agents de sécurité.\\

Les toilettes sèches ont bien fonctionné, à l'exception de temps d’attente un peu élevés dans les moments de forte affluence. Ces temps ont été largement diminués par rapport à ceux de l’édition 2023.\\

La safe zone, bien que spacieuse, n'a pas été pleinement utilisée, peut-être en raison de son emplacement isolé et de son manque d'éclairage.\\

Le bar a fonctionné sans encombre, mais la fréquentation et les ventes ont été inférieures aux prévisions. Plusieurs raisons ont été identifiées : météo, contexte économique, prix des consommations.\\

Les repas végétariens et faits maison pour les équipes du festival (hors artistes) ont été très appréciés, mais leur préparation a représenté une charge de travail considérable, remettant en question leur pérennité sous cette forme.\\

Le retour du voisinage a été globalement positif, mais des améliorations sont envisageables, notamment en matière de communication et de gestion du bruit (orientation des scènes).\\

Il est important de souligner que certaines décisions importantes ont dû être prises rapidement et sans consultation démocratique. Ces éléments méritent d’être étudiés, concernant notamment la répartition des responsabilités et leur impact juridique.

\section{Phase de démontage (19 mai - juin 2024)}

Particulièrement orienté sur la mise en place du festival et son bon déroulement, les organisateurs ont manqué de préparation pour la phase de démontage. Un planning théorique avait été préparé, mais n'a pas été mis à jour ni suivi.\\

Dans la même temporalité, les organisateurs n'ayant pas bénéficié d'un repos suffisant durant l'organisation et la tenue du festival\footnote{une moyenne de 5h de repos par période du 24h, sur les 7 jours précédents a été observée}, une faible productivité a été relevée.

\chapter{Analyse financière}

\section{Aperçu global}
\begin{itemize}
\item Total des dépenses : 150 627,40 €
\item Total des recettes : 150 627,40 €
\item \textbf{Delta\footnote{différence numéraire en dépenses et recettes} : 0,00 €} \textit{(budget équilibré)}
\end{itemize}

\section{Ventilation des dépenses}
\begin{itemize}
\item Création et technique : 76,60 \% du budget total (principalement scènes, technique, artistes)
\item Déplacements, restauration : 20,93 \%
\item Communication : 2,46 \%
\end{itemize}

\section{Ventilation des recettes}
\begin{itemize}
\item FSDIE UB : 34,52 \%
\item Association (fonds propres, recettes bar) : 34,22 \%
\item Autres partenaires publics : 26,28 \%
\item Participation/bénéficiaires : 3,49 \%
\item Partenaires privés : 1,49 \%\\
\end{itemize}

Plusieurs types d'enveloppes peuvent se distinguer : 
\begin{itemize}
\item Financement par l'université de Bordeaux : 71 078,70 €
\item Financement sur enveloppe CVEC\footnote{incluant le financement via le FSDIE} : 69 500 €
\item Financement par de l'argent public\footnote{incluant les financements par l'université de Bordeaux et CVEC} : 91 578,70 € (60.80 \% des recettes)
\end{itemize}

\section{Comparaison du budget entre la 5$^e$ et la 6$^e$ édition}
\subsection{Montant global}
\begin{itemize}
\item Budget réel 2023 : 179 873,87 €, déficit de 36 733,11 € (20,42 \%)
\item Budget réel 2024 : 150 627,40 €, pas de déficit
\item Écart : - 29 246,47 € (soit une diminution de 19,42 \%)
\end{itemize}

\subsection{Analyse par catégorie}
\begin{center}
\begin{table}[h!]
\begin{threeparttable}
\begin{tabular}{|c|c|c|c|c|}
\hline
Catégorie des dépenses & Budget réel 2023 & Budget réel 2024 & Écart & Variation (\%) \\
\hline
Création et technique & 113 323,51 € & 115 387,09 € & + 2 063,58 € & + 1,82 \% \\
Déplacements, restauration & 26 963,06 € & 31 533,15 € & + 4 570,09 € & + 16,95 \% \\
Communication & 30 311,18 €\tnote{a} & 3 707,16 € & - 26 604,02 € & - 87,77 \% \\
Soutien complémentaire\tnote{b} & 9 276,12 € & 0 € & - 9276,12 & - 100 \%\\
\hline
Total & 179 873.87 € & 150 627.40 € & - 29 246.47 € & - 16,26 \%\\
\hline
\end{tabular}
\caption{Tableau comparatif des budgets 2023 et 2024 du festival Mayday}
\begin{tablenotes}
   \item[a] Dont 12 093,76 € pour le feu d'artifice
   \item[a] Soutien exceptionnel par l'université de Bordeaux (surcout afférent à la sécurité)
\end{tablenotes}
\end{threeparttable}
\end{table}
\end{center}

\subsection{Observations}
\begin{itemize}
\item Les dépenses liées à la création et technique ont été légèrement supérieure au budget de 2023, cela s'explique par le passage à 2 jours de fesival et le contexte inflationniste actuel.
\item Les dépenses liées aux déplacements et à la restauration ont été nettement supérieures au budget de 2023, cela correspond aux changements opérés sur la restauration\footnote{Des précisions sont exposées dans la partie \og Innovations de l'édition 2024 \fg{}}.
\item Les dépenses liées à la communication ont été largement inférieures au budget prévu en 2023, en effet la démarche de limitation d'impressions, le changement de prestataire et le fait de ne pas reprogrammer de feu d'artifice ont permis de diminuer dramatiquement ce poste de dépense.\\
\end{itemize}

Le budget 2024 est globalement plus restreint que celui de 2023, avec une baisse significative des dépenses de communication, compensée en partie par une augmentation des dépenses liées aux déplacements et à la restauration (elles-même conséquentes au passage à 2 jours de festival).

\section{Analyse}
\begin{itemize}
\item \textbf{Budget équilibré} : Le budget fait état de dépenses et de recettes égales, ce qui est positif.
\item \textbf{Forte dépendance aux subventions} : 60,80 \% des recettes proviennent de fonds publics (FSDIE, UB, partenaires publics), ce qui peut représenter un risque si ces financements ne sont pas confirmés ou maintenus.
\item \textbf{Coûts de création et technique élevés} : Ces coûts représentent une part importante du budget. Il pourrait être intéressant d'analyser si des économies sont possibles sur ces postes.
\item \textbf{Recettes des buvettes non négligeables} : Les recettes provenant de la vente (bières et limonade) sont importantes et contribuent à l'équilibre du budget (32,23 \%).
\end{itemize}

\section{Recommandations}
\begin{itemize}
\item Diversifier les sources de financement : Réduire la dépendance aux subventions publiques en cherchant d'autres partenaires privés ou en augmentant les recettes propres (par exemple, en développant une activité de conseil ou en organisant des activités payantes),
\item Négocier les coûts de création et technique : Installer une scène principale et une scène annexe dédiée à des DJ sets.
\item Optimiser la communication : Une communication efficace pourrait aider à attirer plus de participants et donc potentiellement à augmenter les recettes ou la réputation du festival\footnote{pouvant attirer davantage de financements}.
\end{itemize}

\section{Conclusion de l'analyse financière}
Le budget final est équilibré, mais présente des risques liés à la dépendance aux subventions publiques et aux coûts élevés de la création et de la technique. En diversifiant les sources de financement et en optimisant certains coûts, il serait possible de renforcer la solidité financière de l'événement.

\chapter{Conclusion}
%\addcontentsline{toc}{chapter}{Conclusion}

Le bilan de cette sixième édition du festival Mayday est globalement positif.\\

L’équipe organisatrice a su relever le défi de la transition vers un festival de deux jours, tout en maintenant un budget équilibré et en proposant une programmation riche et variée.\\

Les innovations mises en place, notamment en matière de restauration et de sensibilisation aux enjeux environnementaux et sociaux, ont été saluées.\\

L’engagement des bénévoles et le soutien des partenaires ont été essentiels à la réussite de l’événement. L'évolution de leur nombre caractérise la renommée grandissante du festival.\\ 

Cependant, des défis subsistent, notamment en ce qui concerne la répartition des tâches au sein de l’équipe, la gestion de la phase de démontage et la communication autour des dispositifs de prévention et des engagements du festival. Ces points devront être améliorés pour les prochaines éditions.\\

Malgré ces défis, le festival Mayday continue de s’affirmer comme un événement majeur de la vie étudiante et culturelle bordelaise, porteur de valeurs d’ouverture, de partage et de responsabilité.\\

L’équipe organisatrice est déterminée à poursuivre ses efforts pour faire de Mayday un festival toujours plus innovant, inclusif et respectueux de l’environnement.

%\chapter*{Annexes}
%\addcontentsline{toc}{chapter}{Annexes}
\listoftables

\end{document}