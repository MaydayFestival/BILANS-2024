\documentclass[12pt,a4paper]{report}
\usepackage[utf8]{inputenc}
\usepackage[french]{babel}
\usepackage[T1]{fontenc}
\usepackage{amsmath}
\usepackage{amsfonts}
\usepackage{amssymb}
\usepackage{graphicx}
\usepackage{lmodern}

\usepackage{geometry}
\geometry{hmargin=2.5cm,vmargin=1.5cm}

\usepackage{soul} \usepackage{color} \newcommand{\hilight}[1]{\colorbox{yellow}{#1}}

\title{Bilan de la 6e édition du Festival Mayday (2024)}
\author{Association M-Tech}
\date{Vendredi 31 mai 2024}

\makeindex

\begin{document}
\maketitle

\begin{abstract}
Le Mayday Festival est une manifestation culturelle. Mêlant musique et sensibilisation aux transitions socio-environnementales, ses objectifs sont :
\begin{itemize}
\item Ouverture de l'accès à la culture à toutes et tous,
\item Sensibilisation du plus grand nombre aux transitions environnementales et sociétales,
\item Être un laboratoire d'incubation, chef de fil en matière d'événementiel éco-responsable,
\item Mettre en avant des artistes locaux, émergents et étudiants
\end{itemize}
Dans le but d'être ouvert à plus de public et de permettre le passage de davantage d'artistes, la 6e édition du festival se déroulait sur 2 jours.Cette évolution ambitieuse, mais cohérente avec les objectifs, fera l'objet d'une analyse dans ce rapport.
Plusieurs périodes ont composé l'organisation et le déroulement du festival. Elles seront détaillées dans le rapport avant de partager le bilan de cette édition.
\end{abstract}

\part{Gestion de projet (août 2023 - avril 2024)}
\chapter{Membres de l'organisation}
\section{Contexte}
Les années marquées par la pandémie "Covid-19" ont eu un fort impact sur la transmission des compétences, connaissances et réseaux des membres d'associations. Le Festival Mayday était porté depuis 2016 par le même noyau dûr, puis repris par une équipe majoritairement novice pour la 5e édition (2023). L'objectif de cette année était notamment de capitaliser sur les connaissances et réseaux restaurés sur les années précédentes. 

\section{Méthodologie}
\section{Membres de l'équipe "Mayday 2024"}

\subsection{subsection}
\subsubsection{subsubsection}
\paragraph{paragraph}
\subparagraph{subparagraph}

\part{Installation et mise en place (10 mai - 17 mai 2024)}

\part{Opérationnel (17 mai - 18 mai 2024)}

\part{Désinstallation et rangement (19 mai - 7 juin 2024)}

\part{Bilan moral}

\part{Bilan financier}
\chapter{Analyse financière - passage à 2 jours}




\end{document}